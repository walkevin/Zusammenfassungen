\section{Kombinatorik}
\begin{tabular}{l >{$}l<{$}}
 Urnenmodell		& \begin{array}[t]{l|l|l}
					& \text{geordnet}	& \text{ungeordnet}\\\hline
		\text{mit Zur�ckl.}	& \begin{array}{llll}
					 A &(1,1)&(1,2)&(1,3)\\
					   &(2,1)&(2,2)&(2,3)\\
					   &(3,1)&(3,2)&(3,3)
					  \end{array}		& \begin{array}{llll}
								 B &(1,1)&(1,2)&(1,3)\\
								   &     &(2,2)&(2,3)\\
								   &     &     &(3,3)\\
								 \end{array}\\\hline
		\text{ohne Zur�ckl.}	& \begin{array}{llll}
					 C&     &(1,2)&(1,3)\\
					  &(2,1)&     &(2,3)\\
					  &(3,1)&(3,2)&
					  \end{array}		& \begin{array}{llll}
								  D&   &(1,2)&(1,3)\\
								   &   &     &(2,3)\\
								   &   &     &     \\
								 \end{array} 
            		  \end{array}\\
			&
			  \begin{array}[t]{llll}
			   \multicolumn{4}{l}{\text{n Kugeln, k Ziehungen}}\\
			   A: n^k		&   B: \binom{n+k-1}{k}	   & C: \frac{n!}{(n-k)!} &   D: \binom{n}{k}
			  \end{array}\\\hline
 Inklusion-Exklusion	& \begin{array}[t]{l}
                    	   |A \cup B| = |A| + |B| - |A \cap B|\\
			   |A \cup B \cup C | = |A| + |B| + |C| - |A \cap B| - |A \cap C| - |B \cap C| + |A \cap B \cap C|\\
			   |A_1 \cup ... \cup A_n| = \sum |A_i| - \sum |A_i \cap A_j| + - ...+(-1)^{n+1}|A_1\cap ... \cap A_n|
                    	  \end{array}\\\hline
 Schubfachprinzip	& \text{Werden $n$ Objekte auf $k<n$ Schubf�cher verteilt,}\\
			& \text{so gibt es ein Schubfach, das mind. 2, resp. $\left\lceil \frac{n}{k} \right\rceil$ Objekte enth�lt}\\
			& \text{Bsp: Von 100 Leuten sind mind. $\left\lceil \frac{100}{12} \right\rceil = 9$ im gleichen Monat geboren.}\\\hline
 Doppeltes Abz�hlen	& |S| = \sum_{a\in A} m_a = \sum_{b\in B} n_b\qquad S = A \times B\\
			&  \begin{array}[t]{ll|l|l|l}
			       & & \multicolumn{3}{l}{\text{Airport}}\\
			       &   & 1 & 2 & 3\\\hline
\multirow{2}{*}{\text{Airline}}& a & 1 & 1 & 0\\
			       & b & 0 & 1 & 1
                   	  \end{array}\\
			& \begin{array}[t]{l}
			   \text{$m_a$ Summe der Zeilen = Summe der angeflogenen Airports einer Airline}\\
			   \text{$n_b$ Summe der Spalten = Summe anfliegenden Airlines eines Airports}\\
			   \text{$S$ = Flugverbindungen}\\
			   \text{$a$ fliegt zum Airport 1 und 2; Airport 2 wird von $a$ und $b$ angeflogen}   
			  \end{array}\\\hline
 Binomialkoeffizient	& \binom{n}{k} = \frac{n!}{k!(n-k)!} = \binom{n}{n-k} = \binom{n-1}{k-1} + \binom{n-1}{k}\quad\text{Pascal-$\bigtriangleup$}\\
			& (x+y)^n = \sum_{k=0}^n \binom{n}{k} x^{n-k}y^k\quad\text{Spez.f�lle: x=y=1; x=-1, y=1}\\\hline
 Vandermonde-Identit�t	& \text{Kugeln in Gruppen unterteilen}\\
			& \binom{n}{k} = \sum_{t=0}^k\binom{r}{t}\cdot\binom{n-r}{k-t}\qquad\text{Bsp: }\binom{2n}{n} = \sum_{k=0}^n \binom{n}{k}\cdot\binom{n}{n-k}\\
\end{tabular}\\
\begin{tabular}{l >{$}l<{$}}
 Permutationen		& \begin{pmatrix}
              		   1 & 2 & 3 & 4 & 5\\
			   3 & 4 & 5 & 2 & 1
              		  \end{pmatrix}
			  \begin{pmatrix}
              		   1 & 2 & 3 & 4 & 5\\
			   3 & 4 & 5 & 2 & 1
              		  \end{pmatrix}^{-1} = 
			  \begin{pmatrix}
              		   3 & 4 & 5 & 2 & 1\\
			   1 & 2 & 3 & 4 & 5
              		  \end{pmatrix}\\
			& \text{Zyklendarstellung: } (1,3,5) \circ (2,4)\\
			& \text{\# ben�tigte Perm. bis zur Identit�t} = \mathrm{kgV} (\text{Einzelzyklen}) \\
			& \text{\# Perm. einer n-Menge} = n!\\
			& \text{\# fixpunktfreie Perm.} = \frac{n!}{e}\\\hline
 Stirling $\bigtriangleup$ 1.Art & S1_{n,k} = S1_{n-1,k-1} + (n-1)\cdot S1_{n-1,k}\\
			    & \text{$S1_{n,k} = $\# Permutationen von n Elementen mit genau k Zyklen}\\
			    & \begin{array}{rccccccccc}
				n=0:&    &    &    &    &  1\\\noalign{\smallskip\smallskip}
				n=1:&    &    &    &  0 &    &  1\\\noalign{\smallskip\smallskip}
				n=2:&    &    &  0 &    &  1 &    &  1\\\noalign{\smallskip\smallskip}
				n=3:&    &  0 &    &  2 &    &  3 &    &  1\\\noalign{\smallskip\smallskip}
				n=4:&  0 &    &  6 &    &  11&    &  6 &    &  1\\\noalign{\smallskip\smallskip}
				\end{array}\\\hline
 Stirling $\bigtriangleup$ 2.Art & S2_{n,k} = S2_{n-1,k-1} + k\cdot S2_{n-1,k}\\
			    & \text{$S2_{n,k} = $\# Partitionen einer n-Menge in k-Mengen}\\
			    & \begin{array}{rccccccccc}
				n=0:&    &    &    &    &  1\\\noalign{\smallskip\smallskip}
				n=1:&    &    &    &  0 &    &  1\\\noalign{\smallskip\smallskip}
				n=2:&    &    &  0 &    &  1 &    &  1\\\noalign{\smallskip\smallskip}
				n=3:&    &  0 &    &  1 &    &  3 &    &  1\\\noalign{\smallskip\smallskip}
				n=4:&  0 &    &  1 &    &  7 &    &  6 &    &  1\\\noalign{\smallskip\smallskip}
				\end{array}\\\hline
 Bell-Zahlen		& B_n = \sum_{k=0}^n S_{n,k}\quad\text{Summe einer Zeile des Stirling $\bigtriangleup$ 2.Art}\\
			& \text{$B_n = $\# �q.rel auf einer n-Menge}\\\hline
 $P_{n,k}$		    & P_{n,k} = \sum_{i=1}^k P_{n-k,i}\\
			    & \text{$P_{n,k} =$\# ungeordnete Partitionen von $n\in\mathbb{N}$ durch $k$ positive Summanden}\\
			    & \text{Bsp: n = 4, k = 2: 4 = 1+3 = 2+2 $\Rightarrow P_{4,2} = 2$}\\
			    & \begin{array}{rccccccccc}
				n=0:&    &    &    &    &  1 &    &  0 &    &  0\\\noalign{\smallskip\smallskip}
				n=1:&    &    &    &  1 &    &  0 &    &  0 &   \\\noalign{\smallskip\smallskip}
				n=2:&    &    &  1 &    &  1 &    &  0 &    &  0\\\noalign{\smallskip\smallskip}
				n=3:&    &  1 &    &  1 &    &  1 &    &   0&   \\\noalign{\smallskip\smallskip}
				n=4:&  1 &    &  2 &    &  1 &    &  1 &    &  0\\\noalign{\smallskip\smallskip}
				\end{array}\\
			    & \text{\# geordnete Partitionen von $n\in\mathbb{N}$ durch $k$ positive Summanden = $\binom{n-1}{k-1}$}
\end{tabular}\\
\begin{tabular}{l >{$}l<{$}}
 Rekursionsgleichung &
\begin{array}[t]{l}
 f_n = f_{n-1} + f_{n-2}\quad f_0 = 0; f_1 = 1\\
\text{Ansatz: } f_n = \lambda^n \Rightarrow \lambda^n = \lambda^{n-1}+\lambda^{n-2}\\
\underbrace{(\lambda^2-\lambda-1)}_{\text{char. Polynom}}\cdot \lambda^{n-2} = 0 \Rightarrow \lambda_{1,2} = \tfrac{1 \pm \sqrt{5}}{2}\\
\text{Lsg: }f_n = a\cdot \lambda_1^n + b\cdot \lambda_2^n\\
\text{Lsg. bei mehrfachen Nst: }f_n = a\cdot \lambda_1^n + b\cdot n \cdot \lambda_1^n + c\cdot n^2\cdot \lambda_1^n\\
\text{Anfangsbedingungen einsetzen: }\\
0 = a\cdot 1 + b \cdot \Rightarrow a= -b\\
1 = a\cdot \lambda_1 - a \cdot \lambda_2 \Rightarrow 1 = a(\lambda_1 - \lambda_2)\\
a = \tfrac{1}{\sqrt{5}}; b = -\tfrac{1}{\sqrt{5}}\\
f_n = \frac{1}{\sqrt{5}}\cdot \left( \left(\frac{1 + \sqrt{5}}{2}\right)^n - \left(\frac{1 - \sqrt{5}}{2}\right)^n \right)
\end{array}
\end{tabular}