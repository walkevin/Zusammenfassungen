\section{Zahlentheorie}
\begin{tabular}{l >{$}l<{$}}
 Teilbarkeit		& a \mid b \Leftrightarrow \exists n: a\cdot n = b \Leftrightarrow \text{ $b$ ist Vielfaches von $a$} \Rightarrow b > a\\
 Euklid.Restsatz	& a = q\cdot m + r\\
 Rest			& r =: R_m(a)\\
 Modulare Arithmetik	& a = b + q\cdot m \Leftrightarrow a \equiv_m b \Leftrightarrow m \mid (a-b)\quad \equiv_m \text{ ist �quivalenzrelation}\\
 ``Resttheoreme``	& \begin{array}[t]{l}
                 	   R_m(a+b) = R_m(R_m(a) + R_m(b))\\
			   R_m(a\cdot b) = R_m(R_m(a) \cdot R_m(b))\\
			   R_m(a^b) = R_m(R_m(a)^b)
                 	  \end{array}\\
 Beispiel		&  \begin{array}[t]{l}
			   R_7(2011^{2011}) = R_7(R_7(2011)^{2011}) = R_7(2^{2011}) =  R_7(2^{3\cdot Q + 1}) = R_7(2^{3\cdot Q} \cdot 2) =\\
			   R_7(R_7(8^Q)\cdot R_7(2)) = R_7(R_7(8)^Q \cdot 2) = R_7(1^Q \cdot 2) = 2
			   \end{array}\\\hline
 Ideal (=ggT)		& \begin{array}[t]{l}
      			   (a,b) = \{x\cdot a + y\cdot b \mid a,b,x,y \in\mathbb{Z}\}\\
			   \forall a_i \in\mathbb{Z} ~\exists d\in\mathbb{Z} : (a_1,...,a_n) = \{z\cdot d \mid z\in\mathbb{Z}\} = d\mathbb{Z}
      			  \end{array}\\
 ggT,kgV		& \begin{array}[t]{l|l}
        		   \text{$d$ heisst ggT von $a$ und $b$, falls:}	& \text{$l$ heisst kgV von $a$ und $b$, falls:}\\\hline
			   d \mid a						& a \mid l\\
			   d \mid b						& b \mid l\\
			   c \mid a \wedge c \mid b \Rightarrow c \mid d	& a \mid m \wedge b\mid m \Rightarrow l\mid m\\
			   \mathrm{ggT}(a,b) = |d| = (a,b)			& a\cdot b = \mathrm{kgV}\cdot\mathrm{ggT}
        		  \end{array}\\
 Erw.Euklid.Algo (EEA)	& \begin{array}[t]{l|l|ll}
               		   \mathrm{ggT}(24,9)	&	& 24				& 9\\\hline
						& 24	& 1				& 0\\
						& 9	& 0				& 1\\
	24 \mod 9 = 24 - \mathbf{2} \cdot 9 =	& 6	& 1-\mathbf{2}\cdot 0 = 1	& 0-\mathbf{2}\cdot 1 = -2\\
		9 - \mathit{1} \cdot 6		& 3	& 0-\mathit{1}\cdot 1 = -1	& 1-\mathit{1}\cdot -2 = 3\\
		6 - 2\cdot 3			& 0	& 1-2\cdot (-1) = 3		& -2 - 2 \cdot 3 = -8\\
			   \multicolumn{3}{l}{\mathrm{ggT}(24,9) = 3 = -24 + 3\cdot 9}\\
               		  \end{array}\\
 Multiplikative Inverse	& \begin{array}[t]{l}
                       	   a\cdot \uuline{x} \equiv_b 1\\
			   \text{Nur vorhanden, falls ggT(a,b) = 1}\\
                       	  \end{array}\\
 Beispiel (mit EEA)	& \begin{array}[t]{l|l|ll}
                   	   \multicolumn{2}{l|}{57^{-1} \equiv_{128} ?}	& 128	& 57\\\hline
							& 128		& 1	& 0\\
							& 57		& 0	& 1\\
						2	& 14		& 1	& -2\\
						4	& 1		& -4	& \uuline{9}\\
			   \multicolumn{4}{l}{-4\cdot 128 + 9\cdot 57 = 1 \Rightarrow \uuline{9}\cdot 57 \equiv_{128} 1}
                   	  \end{array}\\
\end{tabular}\\
\begin{tabular}{l >{$}l<{$}}
 Chin. Restsatz (CRS)		& \begin{array}[t]{l}
                            	   \begin{array}{l|ll}
				    x \equiv_3 1	& m_1 = 3	& a_1 = 1\quad M = 3\cdot 4\cdot 5 = 60\\
				    x \equiv_4 3	& m_2 = 4	& a_2 = 3\\
				    x \equiv_5 3	& m_3 = 5	& a_3 = 3\\
				   \end{array}\\
				  M_i = \frac{M}{m_i}:\quad M_1 = 20,~M_2 = 15,~M_3 = 12\\
				  M_iN_i \equiv_{m_i} 1\text{ erraten}: \begin{array}{l}
									  20\cdot N_1 \equiv_3 1 \rightarrow N_1 = 2\\
									  15\cdot N_2 \equiv_3 1 \rightarrow N_2 = 3\\
									  12\cdot N_3 \equiv_3 1 \rightarrow N_3 = 3\\
									\end{array}\\
				  \tilde{x} = R_M\left(\sum_{i=1}^{r} a_iM_iN_i\right)\\
				  \tilde{x} = R_{60}(1\cdot 20 \cdot 2 + 3 \cdot 15 \cdot 3 + 3 \cdot 12 \cdot 3 ) = R_{60} (40+135+108) = R_{60}(283) = 43\\
				   x = \tilde{x} (\mod 60)
                            	  \end{array}\\
\end{tabular}\\
Umwandlung in ein regul�res CRS-System aus CRS-System mit nicht teilerfremden Moduli\\
$
\left. \begin{array}{ll}
        z\equiv_6 1	& \begin{cases}
			   z\equiv_2 1\\
                   	   z\equiv_3 1
                   	  \end{cases}\\
	z\equiv_{10} 3	& \begin{cases}
			   z\equiv_2 3 &= z\equiv_2 1\\
                   	   z\equiv_5 3
                   	  \end{cases}\\
        z\equiv_{75} 28	& \begin{cases}
			   z\equiv_3 28 &= z\equiv_3 1\\
                   	   z\equiv_{25} 28 &= z\equiv_{25} 3
                   	  \end{cases}\\
       \end{array}
\right\rbrace
\begin{array}{l}
	\left. \begin{array}{l}
		z\equiv_2 1\\
		z\equiv_3 1
		\end{array}
	\right\rbrace z\equiv_6 1\\
	z\equiv_{25} 3
\end{array}
\Rightarrow
\begin{vmatrix}
 z\equiv_6 1\\
 z\equiv_{25} 3
\end{vmatrix}
z = 103 (\mod 150)
$\\
\section{Algebra}
\begin{tabular}{l >{$}l<{$}}
Gruppe (G,$\ast$)	& \begin{array}{ll}
                 	   \text{Assoziativit�t}	& \forall a,b,c \in G: (a\ast b)\ast c = a\ast(b\ast c)\\
			   \text{Neutrales Element}	& \exists e: a \ast e = e \ast = a \forall a\\
			   \text{Inverses}		& \forall a \exists b : a \ast b = b\ast a = e
                 	  \end{array}\\
Geometrische Gruppen	& \text{Rotationen, �hnlichkeitstrafos, Translationen, Spiegelungen}\\
Symmetriegruppe $s_n$	& \text{Spiegelungen und Rotationen am n-Eck}\quad |s_n| = 2n\\
Untegruppen		& H \subset G \text{ falls $H$ bzgl. $\ast$ selbst eine Gruppe ist}\\
Lagrange		& |H| \mid |G|\\
			& \text{Falls G Untergruppen hat, so sind diese paarweise disjunkt}\\
Produkt von Gruppen	& \mathbb{Z}_{15}^{\ast} = \left\lbrace \tbinom{1}{1},\tbinom{2}{1},\tbinom{3}{1},\tbinom{4}{1},\tbinom{1}{2},\tbinom{2}{2},\tbinom{3}{2},\tbinom{4}{2}\right\rbrace
 = \{1,2,3,4\} \times \{1,2\} = \mathbb{Z}_5^{\ast} \times \mathbb{Z}_3^{\ast}\\
			& \mathbb{Z}_{15}^{\ast} = \{1,2,4,7,8,11,13,14\}\\
			& \mathbb{Z}_m = \mathbb{Z}_p \times \mathbb{Z}_q\quad\text{falls $m = p\cdot q$}\\
Endliche Gruppen	& \mathbb{Z}_p\quad  \mathbb{Z}_{p-1}^{\ast}\quad p\text{ prim}\\
Ordnung endl. Gruppen	& \begin{array}[t]{l}
                     	    \mathrm{ord}(a) := \min\{i > 0 \mid a^i = e\}\\
			    \mathrm{ord}(a) \mid |G| \quad a^{|G|} = e
                     	  \end{array}\\
\end{tabular}\\
\begin{tabular}{l >{$}l<{$}}
Zyklische endl. Gruppen	& \begin{array}[t]{l}
                       	   G \text{ heisst zyklisch, falls }\exists g \in G: \mathrm{ord}(g) = |G|\\
			   G = \{g^0, g^1, g^2,...,g^{|G|-1}\}\\
			   g \text{ heisst Generator von G, } G = <g>\\
			   |G| \text{ prim} \Rightarrow G \text{ zyklisch} \Rightarrow G\text{ abelsch}\\
			   |G| \text{ prim} \Rightarrow \text{jedes $g \neq e$ ist Generator}
                       	  \end{array}\\
Beispiel		& \mathbb{Z}_p^{\ast}\\
$\varphi$-Funktion	& \begin{array}[t]{l}
                  	   \mathbb{Z}_m^{\ast} := \{ a \in \mathbb{Z}_m | \mathrm{ggT}(a,m) = 1\}\\
			   \varphi(m) = |\mathbb{Z}_m^{\ast}|\quad\text{''Anzahl teilerfremde Zahlen von m``}\\
			   \varphi(m) = \prod_{i=1}^r (p_i-1)\cdot p_i^{e_i-1}\quad m = \prod_{i=1}^r p_i^{e_i}\quad\text{Primfaktorzerlegung}\\
			   \varphi(p) = p-1
                  	  \end{array}\\
Beispiel		& \begin{array}[t]{l}
        		   |\mathbb{Z}_{45}^{\ast}|\quad 45=3^2\cdot 5^1\Rightarrow p_1 = 3,e_1 = 2;p_2 = 5,e_2 = 1\\
			   \varphi(45) = (3-1)\cdot 3^{2-1}\cdot (5-1)\cdot 5^{1-1} = 2\cdot 3\cdot 4 = 24
        		  \end{array}\\
Satz von Fermat-Euler	& \begin{array}[t]{l}
                     	   \forall m \geq 2 ~\forall a: \mathrm{ggT}(a,m) = 1\\
			   a^{\varphi(m)}\equiv_m 1\\
			   a^{p-1}\equiv_p 1
                     	  \end{array}\\
Diskreter Logarithmus	& \begin{array}[t]{l}
                     	   R_p(a^x) \leftrightarrow x \xleftrightarrow{isomorph} \mathbb{Z}_p^{\ast}\\
			   x \rightarrow R_p(a^x)\quad\text{ einfach}\\
			   R_p(a^x) \rightarrow x\quad\text{ schwierig, l�sbar z.B. mit Babystep-Giantstep}
                     	  \end{array}\\
Babystep-Giantstep Algo	& \begin{array}[t]{l}
                       	   \text{Eingabe: zykl.endl.Gruppe $G$, Generator $g$, Gruppenelement $a$}\\
			   \text{Ausgabe: }x = \log_g a\\
			   m := \left\lceil \sqrt{|G|} \right\rceil\\
			   \forall j \in \{0,...,m-1\} \text{ berechne $g^j$ und speichere $(j,g^j)$ in der Tabelle T}\\
			   \forall i \in \{0,...,m-1\} \text{ berechne $a\cdot \left(g^{-m}\right)^i$ und suche den Wert in T}\\
			   \text{Falls gefunden, gib $im+j$ aus}
                       	  \end{array}\\
Beispiel		& \begin{array}[t]{l}
        		   R_{29}(11^x) = 3\quad G = 29, g = 11, a = 3\\
			   |G| = \varphi(29) = 28 \Rightarrow m:= \left\lceil \sqrt{28} \right\rceil = 6\\
			   \begin{array}{l|l|l|l|l|l|l}
			    j	& 0 & 1  & 2 & 3  & 4  & 5\\\hline
			    11^j& 1 & 11 & 5 & 26 & 25 & 14
			   \end{array}\\
			   11^{-6} = 11^{28-6} = 13\\
			   \begin{array}{l|l|l|l}
			   i		& 0 & 1  & 2\\\hline
			   3\cdot 13^i	& 3 & 10 & 14
			   \end{array}\\
			   i = 2, j = 5 \Rightarrow x = 2\cdot 6 + 5 = 17
        		  \end{array}\\
\end{tabular}\\
\begin{tabular}{l >{$}l<{$}}
K�rper (K,+,$\cdot$)	& \begin{array}[t]{l}
                    	   (K,+) \text{ abelsche Gruppe bzgl. Addition mit NE 0}\\
			   (K^{\ast},\cdot) \text{ abelsche Gruppe bzgl. Multiplikation mit NE 1}\\
			   \forall a,b,c \in K:~a\cdot(b+c) = a\cdot b + a \cdot c
                    	  \end{array}\\
Endliche K�rper		& \begin{array}[t]{l}
               		   (\mathbb{Z}_p,+,\cdot) =: GF(m)\\
			   \text{$p$ prim, $n\in\mathbb{N} \Rightarrow \exists !$ endlicher K�rper mit $p^n$ Elementen, genannt $GF(p^n)$}\\
			   \text{Werden Zeilen oder Spalten eines GF vertauscht,}\\
			   \text{so ist es immer noch der gleiche (isomorphe) GF}
               		  \end{array}\\
Beispiele		& \begin{array}[t]{l|ll||l|ll}
         		   \multicolumn{6}{l}{GF(2)}\\
			   + & 0 & 1 & \cdot & 0 & 1\\\hline
			   0 & 0 & 1 &     0 & 0 & 0\\
			   1 & 1 & 0 &     1 & 0 & 1
         		  \end{array}\quad
			  \begin{array}[t]{l|llll||l|llll}
         		   \multicolumn{10}{l}{GF(4) = GF(2^2)}\\
			   + & 0 & 1 & a & b & \cdot & 0 & 1 & a & b\\\hline
			   0 & 0 & 1 & a & b &     0 & 0 & 0 & 0 & 0\\
			   1 & 1 & 0 & b & a &     1 & 0 & 1 & a & b\\
			   a & a & b & 0 & 1 &     a & 0 & a & b & 1\\
			   b & b & a & 1 & 0 &     b & 0 & b & 1 & a
         		  \end{array}\\
irreduzible Polynome	& \begin{array}[t]{l}
			   \text{Analogon zu Primzahlen}\\
                    	   \text{Polynom hat Nullstellen $\Rightarrow$ Nicht irreduzibel}\\
			   \text{F�r Polynome mit Grad $\leq 3$ gilt obiges in beide Richtungen.}
                    	  \end{array}\\
iP �ber GF(2)		& \begin{array}[t]{ll}
             		   \text{Grad 1:} & x, ~x+1\\
			   \text{Grad 2:} & x^2+x+1\\
			   \text{Grad 3:} & x^3+x+1, ~x^3+x^2+1\\
			   \text{Grad 4:} & x^4+x^3+1, ~x^4+x+1, ~x^4+x^3+x^2+x+1
             		  \end{array}\\
Elemente des $GF(2^3)$		& \begin{array}[t]{ll}
				   \text{Elemente des $GF(2): 0,1$}\\
                          	   \text{Nimm ein irreduzibles Polynom, z.B. $P = x^3+x+1=0$}\\
				   \text{Alle Polynome, die strikt kleiner sind als das Primpolynom, sind $\in GF(2^3)$}
				   \Rightarrow 0,1,x,x+1,x^2,x^2+1,x^2+x,x^2+x+1
                          	  \end{array}\\
 Teilbarkeit		& a(x) \mid b(x) \Leftrightarrow \exists n(x): a\cdot n(x) = b \Rightarrow \mathrm{deg}(b) \geq \mathrm{deg}(a)\\
 Euklid.Restsatz	& a(x) = q(x)\cdot m(x) + r(x)\\
 Rest			& r(x) =: R_{m(x)}(a(x))\\
 Modulare Arithmetik	& a(x) = b(x) + q(x)\cdot m(x) \Leftrightarrow a(x) \equiv_{m(x)} b(x) \Leftrightarrow m(x) \mid (a(x)-b(x))\\
 Multiplikative Inverse & a(x) \cdot \uuline{z(x)} \equiv_{m(x)} 1\\
 Beispiel (mit EEA)	& \begin{array}[t]{l|l|ll}
                   	   \multicolumn{2}{l|}{x^{-1} \equiv_{x^2+x+1}}	& x^2+x+1	& x\\\hline
							& x^2+x+1	& 1		& 0\\
							& x		& 0		& 1\\
						x+1	& 1		& 1		& \uuline{x+1}\\
			   \multicolumn{4}{l}{(x+1)\cdot x + x^2+x+1 \equiv_{x^2+x+1} 1 \Rightarrow \uuline{x+1}\cdot x \equiv_{x^2+x+1} 1}
                   	  \end{array}\\
\end{tabular}\\
\begin{tabular}{l>{$}l<{$}}
RSA	& \begin{array}[t]{lll}
   	   \text{Alice}					&\text{Eve}		& \text{Bob}\\\hline
	   p,q := \text{prim}				&			& m:=\text{Nachricht}\quad m\leq n\\
	   \rightarrow n =p\cdot q			&			& \\
	   \rightarrow f = \varphi(n) = (p-1)(q-1)	&			& \\
	   e:= \mathrm{ggT}(e,f) = 1			&\xRightarrow{~~n,e~~}	& c = R_n(m^e)\\
	   \rightarrow d = q\cdot f+e^{-1}		&\xLeftarrow{~~~c~~~}		& \\
	   \rightarrow m = R_n(c^d)			&			& 	   
   	  \end{array}\\
	& \begin{array}[t]{l}
	   \text{Angriffspunkte}\\\hline
	   \text{1. Finde $n = p\cdot q$, berechne $d$ und $m$ wie Alice}\\
	   \text{2. $c = R_n(m^e)$ durchprobieren}
	  \end{array}\\
Diffie-Hellman	& \begin{array}[t]{lcl}
   	   \text{Alice}					&\text{Eve}			& \text{Bob}\\\hline
	   				&\text{Einwegfkt: $x\mapsto R_p(g^x)$}		& \\
	   p := \text{prim, }g:= \text{Generator}	&\xRightarrow{~~~p,g~~~}		& \\
	   x:=\text{geheim} \in \{0,...,p-2\}		&\xRightarrow{~~R_p(g^x)~~}	& \\
							&\xLeftarrow{~~R_p(g^y)~~}	& y:=\text{geheim} \in \{0,...,p-2\}\\
	   \rightarrow K_{AB} = R_p\left((g^y)^x\right)	&				& \rightarrow K_{BA} = R_p\left((g^x)^y\right)\\
	   \multicolumn{3}{l}{K_{AB} = K_{BA} \text{ ist der gemeinsame Schl�ssel}}
   	  \end{array}\\ 
\end{tabular}\\


