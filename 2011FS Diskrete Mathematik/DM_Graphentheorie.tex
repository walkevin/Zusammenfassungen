\section{Graphentheorie}
\begin{tabular}{l >{$}l<{$}}
Graph		& G = (V,E)\quad\text{V = Knoten, E = Kanten}\\
		& \sum_i \deg(v_i) = 2|E|\\
Kreis		& \text{Sei $(v_i,...,v_l)$ paarweise verschiedene Knoten, $l\geq 3$}\\
		& \text{$G$ Kreis :}\forall i = 1,...,l-1: (v_i,v_{i+1})\in E \wedge (v_l,v_1)\in E\\
zsh.		& \text{$G$ zusammenh�ngend. :}\forall u,v \in V \exists \text{u-v-Pfad}\\
		& \text{$G=(V,E)$ hat mind. $|V|-|E|$ Zsh.komp.}\\
		& \text{$G$ zsh. $\Rightarrow |V|-|E| \leq 1$}\\
Teilgraph	& \text{$H = (V',E')$ Teilgraph von $G = (V,E)$}: V'\subset V \wedge E'\subset E \wedge E' \subset V'\times V'\\
Br�cke		& e \in E \text{ Br�cke: $G' = (V,E\smallsetminus{e})$ hat eine Zsh.komp. mehr als $G = (V,E,)$}\\
Baum		& \begin{array}[t]{ll}
		        &\text{kreislos, zsh., $|V| \geq 2$, mind. 2 Bl�tter}\\
			& \text{$G$ ist ein Baum}\\
\Leftrightarrow		& \text{$G$ ist zsh. $\wedge |V| = |E| + 1$}\\
\Leftrightarrow		& \text{$G$ ist kreislos $\wedge |V| = |E| + 1$}\\
\Leftrightarrow		& \text{$G$ ist zsh. $\wedge$ jede Kante ist Br�cke}\\
\Leftrightarrow		& \text{$G$ ist maximal kreislos (zus�tzl. Kante ergibt Kreis)}\\
\Leftrightarrow		& \text{$\forall u,v \in V \exists!$ u-v-Pfad$\qquad \exists:$ zsh., !: Kreislosigkeit}
    		  \end{array}\\
Spannbaum	& \text{Sei $G=(V,E)$ Baum. $H=(V,E')$ Spannbaum: $H$ Baum $\wedge E'\subset E$}\\
		& \text{\# Kanten im Spannbaum = $|V|-1$}\\
$K_n$		& \text{$K_n$: vollst�ndiger Graph~~~~\# Kanten im $K_n = \binom{n}{2} = \frac{n(n-1)}{2}$}\\
Cayley		& \text{\# Spannb�ume im }K_n = n^{n-2}\\
$C_n$		& \text{$C_n$: Kreis(Zyklen) der L�nge $n, n\geq 3$}\\
$M_{m,n}$	& \text{$M_{m,n}$: Gittergraph, Meshgraph}\\
$K_{m,n}$	& \text{$K_{m,n}$: vollst. bipartiter (zweif�rbbarer) Graph}\\
$Q_d$		& \text{$Q_d$: d-dim. Hyperkubus}\\
		& V = {0,1}^d\quad {u,v} \in E:\Leftrightarrow d_H(u,v) = 1\quad\text{$d_H$: Hammingdistanz}\\
		& |E| = 2^{d-1}\cdot d\quad |V| = 2^d\\
\end{tabular}\\
\begin{tabular}{l >{$}l<{$}}
Eulertour	& \text{Kreis, der jede Kante genau einmal enth�lt}\\
		& \text{Ein zsh. Graph hat eine {\itshape geschlossene} Eulertour gdw. alle Knotengrade gerade sind.}\\
		& \text{~~Bsp: $K_n$ f�r $n$ ungerade, $Q_d$ f�r $d$ gerade}\\
		& \text{Ein zsh. Graph hat eine {\itshape offene} Eulertour gdw. zwei Knotengrade ungerade sind}\\\hline
Hamiltonkreis	& \text{Kreis, der jeden Knoten genau einmal enth�lt}\\
		& \text{$M_{m,n}$ hamiltonsch, falls $m\cdot n$ gerade und $n\geq 2$}\\
		& \text{$Q_d$ hamiltonsch f�r $d\geq 2$}\\
		& \text{$G=(V,E)$ mit $\deg(V) \geq \frac{|V|}{2} \forall v \wedge |V| \geq 3\Rightarrow G$ hamiltonsch}\\\hline
Planare Graphen & \text{$G=(V,E)$ planar, wenn er ohne Kanten�berschneidung gezeichnet werden kann.}\\
		& \begin{array}[t]{ll}
		   \text{$G$ planar, zsh.}	& \Rightarrow |V| + f - |E| = 2\\
		   \multicolumn{2}{l}{\text{$\quad G$ teilt die Ebene in $f$ Gebiete}}\\
		   \text{$G$ planar, $|V| \geq 3$}	& \Rightarrow |E| \leq 3|V|-6 \wedge \overline{\deg(v)} < 6\\
		   \text{$G$ planar, bipartit}		& \Rightarrow |E| \leq 2|V|-4 \wedge \overline{\deg(v)} < 4\\
		   \multicolumn{2}{l}{\text{Achtung: Die Umkehrrichtung gilt jeweils nicht!}}
		  \end{array}\\
		& \begin{array}[t]{l}
		   \text{Nicht planar sind (vollst�ndige Liste nach Kuratowski):}\\
		   K_5, K_{3,3}	\qquad K_6, K_{3,4}, K_{4,4},...\\
		   \text{Graphen, die den $K_5$ oder $K_{3,3}$ als Teilgraph enthalten}\\
		   \text{Unterteilte Graphen von $K_5$ oder $K_{3,3}$}\\
		   \text{Graphen, die diese unterteilte Graphen als Teilgraph enthalten}
		  \end{array}\\\hline
Knotenf�rbbarkeit & \text{$G=(V,E)$ zsh. $|V| \geq 2$, zweif�rbbar = bipartit ($\chi(G) = 2)$}\\
		  & \text{gdw er keinen Kreis ungerader L�nge enth�lt}\\
		  & G=(V,E) \text{planar} \Rightarrow \chi(G) \leq 4\\
		  & \begin{array}[t]{lll}
		     \chi(K_n) = n	& \chi(K_{m,n}) = 2	& \chi(M_{m,n}) = 2\\
		     \chi(Q_d) = 2	& \chi(\text{Baum}) = 2	& \chi(C_n) = \begin{cases}
		                  	                	               2	& n \text{ gerade}\\
		                  	                	               3	& n \text{ ungerade}\\
		                  	                	              \end{cases}
		    \end{array}
\end{tabular}
