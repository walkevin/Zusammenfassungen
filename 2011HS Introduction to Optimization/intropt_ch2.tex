\section{LP Basics I}
\subsection{The Dual}
\subsubsection{Rationale for the dual}
{\itshape Ref. P.16-17}\\
A strategy to prove the optimality of a LP is to make a linear combination of the constraints such that the summed constraint ``lower-equals'' the objective function. The right hand side of the summed constraint will be the optimal solution.\\
{\itshape Example:}\\
$
\begin{array}{lrcrr}
\max		& 6x_1	& + & x_2\\
\text{s.t.}\\
\text{E1}	& 2x_1	& + 	 & x_2 	& \leq 4\\
\text{E2}	&  x_1	&   	 &	& \leq 3\\
\text{E3}	&  x_1	& \geq 0 & x_2 \geq 0
\end{array}\\\\
\text{Summed constraint: } 1\cdot E1 + 4 \cdot E2 = 6x_1 + x_2 \leq 16\\
$
That means that our objective value cannot exceed 16, but it still may be smaller. In fact, $6x_1 + x_2 \leq 3\cdot E1 + 0\cdot E2 = 6x_1 + 3x_2 \leq 12$ is the smallest upper bound.\\\\
In general, we might not know or see the coefficients for the linear combination easily. Thus we formulate a problem to find them, which turns out to be a LP again. As above, we can't calculate the solution directly, instead we overestimate it, aiming to find the smallest upper bound for the solution.\\
Consider the LP
$
\begin{array}[t]{lrcrr}
\max		& c_1x_1  & + & c_2x_2\\
\text{s.t.}\\
\text{E1:}	& a_{11}x_1 & + & a_{12}x_2	& \leq b_1\\
\text{E2:}	& a_{21}x_1 & + & a_{22}x_2 	& \leq b_2\\
\text{E3:}	& a_{31}x_1 & + & a_{32}x_2	& \leq b_3\\
\text{E4:}	& x_1	& \geq 0 & x_2 \geq 0
\end{array}\\
$
In the following $y_i$ denote the coefficients we want to find. Linear combination of the constraints yields:\\
\begin{equation*}
 y_1 \cdot E_1 + y_2 \cdot E_2 + y_3 \cdot E_3=
(a_{11}y_1 + a_{21}y_2 + a_{31}y_3) x_1 + (a_{12}y_1 + a_{22}y_2 + a_{32}y_3) x_2 \leq y_1b_1 + y_2b_2 + y_3b_3
\end{equation*}
Now we overestimate the solution: (Note that $E_4$ must hold in order the following be an over- and not an underestimation)
\begin{equation*}
\begin{array}{ll}
(a_{11}y_1 + a_{21}y_2 + a_{31}y_3)	& \geq c_1\\
(a_{12}y_1 + a_{22}y_2 + a_{32}y_3)	& \geq c_2
\end{array}
\end{equation*}
To make the overestimation smallest possible, we have a minimization problem, therefore the dual problem is:\\
\textbf{Dual problem:}\quad

$
\begin{array}[t]{lrcrcrr}
\min		& y_1b_1  & + & y_2b_2	& +	& y_3b_3\\
\text{s.t.}	& a_{11}y_1 & + & a_{21}y_2 & + & a_{31}y_3	& \geq c_1\\
		& a_{12}y_1 & + & a_{22}y_2 & + & a_{32}y_3	& \geq c_2\\
		& y_i \geq 0
\end{array}
$


\subsubsection{Forms of Dual LP}
{\itshape Ref. P.22}\\
Note two equivalences of linear constraints\\
\begin{enumerate}
 \item $a^Tx = b \Leftrightarrow a^Tx \leq b \text{ AND } -a^Tx \leq -b$
 \item $x_j \text{ free} \Leftrightarrow x_j = x'_j - x_j'',\text{where }x'_j, x_j'' \geq 0$
\end{enumerate}
$
\begin{array}[t]{p{6cm}p{6cm}}
  \begin{array}[t]{ll}
    \max	& c^Tx\\
    \text{s.t.}	& Ax \leq b\\
		& x \geq 0
  \end{array}
&
  \begin{array}[t]{ll}
    \min	& b^Ty\\
    \text{s.t.}	& A^Ty \geq c\\
		& y \geq 0
  \end{array}\\\hline
%Zeile 2
  \begin{array}[t]{ll}
    \max	& c^Tx\\
    \text{s.t.}	& Ax = b\\
		& x \geq 0
  \end{array}
&
  \begin{array}[t]{ll}
    \min	& b^Ty\\
    \text{s.t.}	& A^Ty \geq c\\
		& y \text{ free}
  \end{array}\\\hline
%Zeile 3
  \begin{array}[t]{ll}
    \max	& c^Tx\\
    \text{s.t.}	& Ax \leq b\\
		& x \text{ free}
  \end{array}
&
  \begin{array}[t]{ll}
    \min	& b^Ty\\
    \text{s.t.}	& A^Ty = c\\
		& y \geq 0
  \end{array}\\
\end{array}
$
\\
To prove these alternative forms of Dual LP, use the equivalences 1 and 2, and then try to transform it to the standard primal-dual problem in row 1 of this table (i.e. try to transform equalities into inequalities, and free variables into variables $\geq 0$\\
\subsection{Weak and Strong Duality}
{\itshape Ref. P.18-19}\\
\begin{tabular}{p{4cm} >{$}p{15cm}<{$}}
{\bfseries Weak duality}	& \text{For any pair of primal and dual feasible solutions $x$ and $y$: } \mathbf{c^Tx \leq b^Ty}\\
{\itshape Proof}		& \text{{\itshape Ref. Lecture, Blackboard}}\\
				& \text{Let $(x,y)$ be feasible solutions of a primal LP and its corresponding dual.}\\
				& \text{Therefore }\begin{array}{llll}
     		                   1. & A\bar{x} \leq b	& 3. & \bar{y} \geq 0\\
				   2. & \bar{x} \geq 0	& 4. & A^T\bar{y} \geq c
     		                  \end{array}\\
				& \text{Then } b^Ty \underbrace{\geq}_{1.,3.} (Ax)^Ty = x^TA^Ty \underbrace{\geq}_{4.,2.} x^Tc = c^Tx\\
Remark II			& \begin{array}[t]{l}
				    \text{If dual is unbounded / infeasible $\Rightarrow$ primal is infeasible / unbounded, because}\\
				    \text{Weak Duality: }c^Tx \leq b^Ty;\enspace y \rightarrow -\infty
				  \end{array}\\
{\bfseries Strong duality}	& \text{If an LP has an optimal solution $\bar{x}$, then the dual has an optimal solution $\bar{y}$, and }\\
				& \mathbf{c^T\bar{x} = b^T\bar{y}}\\
Remark I {\itshape Proof for ``$\Leftarrow$''}	& \text{If $\exists \bar x, \bar y$ s.t. $c^T\bar{x} = b^T\bar{y},$ then $\bar{x}$ and $\bar{y}$ are both optimal} $\par$
				\text{as by Weak duality $c^Tx \leq b^Ty ~\forall y$, thus $\max c^Tx \leq b^Ty \Leftrightarrow c^Tx = b^Ty$}
\end{tabular}
\subsection{Complementary Slackness Conditions}
{\itshape Ref. P.19}\\
The following conditions are equivalent for a dual pair of solutions $\bar{x}$ and $\bar{y}$\\
\begin{tabular}{p{4cm} >{$}p{16cm}<{$}}
(a) 	& \text{$\bar{x}$ and $\bar{y}$ are optimal solutions}\\
(b)  	& c^T\bar{x} = b^T\bar{y}\\
(c)  	& \bar{y}^T(b-A\bar{x}) = 0 \wedge \bar{x}^T(A^T\bar{y} -c) = 0\\
(c') 	& \bar{y_i}(b-A\bar{x})_i = 0 ~\forall i \wedge \bar{x_j}^T(A^T\bar{y} -c)_j = 0 ~\forall j\\
(c'')	& \bar{y}_i > 0 \Rightarrow (b-A\bar{x})_i = 0 \forall i \wedge \bar{x}_j > 0 \Rightarrow (A^T\bar{y} -c)_j = 0~\forall j
\end{tabular}\\\\
{\itshape Proof:}\\
\begin{tabular}{p{4cm} >{$}p{16cm}<{$}}
(b) $\Rightarrow$ (a)	& \text{by weak duality}\\
(a) $\Rightarrow$ (b)	& \text{by strong duality}\\
(b) $\Leftrightarrow$ (c') 	& \text{Let $(\bar{x},\bar{y})$ be feasible solutions.}\\
				& \text{Therefore }\begin{array}{llll}
     		                   1. & A\bar{x} \leq b	& 3. & \bar{y} \geq 0\\
				   2. & \bar{x} \geq 0	& 4. & A^T\bar{y} \geq c
     		                  \end{array}\\
			& \begin{array}[t]{ll}
			    0 \underbrace{=}_{(b)} b^T\bar{y}-c^T\bar{x}	&= b^T\bar{y} - \underbrace{\bar{y}^T A\bar{x}}_{\text{Skalar}} + \bar{y}^T A\bar{x} - c^T\bar{x}\\
										&= \bar{y}^T(b-A\bar{x}) + \bar{x}^T (A^T\bar{y}-c)\\
										&= \sum_{i=1}^m \underbrace{\bar{y}_i}_{3.) \geq 0} \underbrace{(b-A\bar{x})_i}_{1.) \geq 0}+ \sum_{i=1}^n \underbrace{\bar{x}_j}_{2.) \geq 0} \underbrace{(A^T\bar{y} - c)_j}_{4.) \geq 0} = 0\\
			   \multicolumn{2}{l}{\text{Therefore, } \bar{y}_i(b-A\bar{x})_i = 0 \wedge \bar{x}_j(A^T\bar{y} - c)_j = 0\quad\text{(c')}}
			  \end{array}
\end{tabular}

\subsection{Recognition of Infeasibility}
{\itshape Ref. P.19-20}\\
In principle, we just have to find two contradictory constraints or linear combinations of constraints. After Farkas' Lemma, this kind of proof is always possible.\\
\begin{tabular}{p{4cm} >{$}p{16cm}<{$}}
 \textbf{Farkas' Lemma:}	& \text{A System of inequalities }Ax \leq b, x \geq \mathbf{0} \text{ is {\bfseries infeasible}}\Leftrightarrow \exists y \geq \mathbf{0}: A^Ty \geq \mathbf{0} \wedge b^Ty < 0\\
 {\itshape Proof} for $\Rightarrow$		& Ax \leq b \underbrace{\Rightarrow}_{y \geq 0} \underbrace{y^TA}_{\geq \mathbf{0}^T}x \nleq \underbrace{y^Tb}_{< 0}~\forall x
\end{tabular}\\
\subsection{Recognition of Unboundedness}
{\itshape Ref. P.20-21}\\
If there is a vector $z$, which can be multiplied by any $\alpha > 0$ and added to a feasible solution $x$, (i.e. $x' = x+\alpha z$), the LP is unbounded. Add a graphic (P.21)\\
\begin{tabular}{p{4cm} >{$}p{16cm}<{$}}
\textbf{Theorem 2.5}	& \text{A LP $\max c^Tx$, s.t. $Ax \leq b, x \geq 0$ is unbounded if and only if it has a}\\
			& \text{feasible solution $x$ and there exists (a direction) $z$ such that $z\geq 0, Az \leq 0, c^Tz > 0$}\\
{\itshape Proof}	& \begin{array}[t]{lll}
			   \multicolumn{3}{l}{\text{Let }x' = x+\underbrace{\alpha}_{> 0} z}\\
     			   \text{Because of...}					&		& \text{this must hold}\\
			   x'  \geq 0						& \Rightarrow	& z \geq 0\\
			   Ax' \leq b \Leftrightarrow Ax + A\alpha z \leq b	& \Rightarrow	& A\alpha z \leq 0 \Rightarrow Az \leq 0\\
			   c^Tx' > c^Tx \Leftrightarrow c^Tx+c^T\alpha z) > c^Tx& \Rightarrow	& c^T\alpha z > 0 \Rightarrow c^Tz > 0
     			  \end{array}
\end{tabular}\\
Important! $x$ must be a \textbf{feasible} solution!\\

\subsection{Coexistent Primal and Dual}
\begin{tabular}{lr|ccc|}
			&\multicolumn{1}{l}{}	& \multicolumn{3}{c}{Dual}\\
			&\multicolumn{1}{l}{}	& optimal 		& infeasible		& \multicolumn{1}{c}{unbounded}\\\cline{3-5}
\multirow{3}{*}{Primal}	& optimal		& \checkmark, s.D.	& \texttimes, s.D.	& \texttimes, w.D.\\
			& infeasible		& \texttimes, s.D.	& \checkmark		& \checkmark, w.D.\\
			& unbounded		& \texttimes, w.D.	& \checkmark, w.D.	& \texttimes, w.D.\\\cline{3-5}
\end{tabular}\\\\
s.D. stands for ``proven by strong duality'', w.D. stands for ``proven by weak duality''. A LP which is primal infeasible and dual infeasible can be constructed as follows:\\\\
$
\begin{array}{lrcccl}
\text{Primal:}	& \max		& x_1\\
		& \text{s.t.}	& 	& x_2	& \leq	& -1\\
		&		& -x_1	&	& \leq	& 0\\
		&		&	&x_1,x_2& \geq	& 0\\
\multicolumn{6}{l}{\text{Infeasible, because $-1 \geq x_2 \geq 0$ contradictory}}
\end{array}
\begin{array}{lrcccl}
\text{Dual:}	& \min		& -y_1\\
		& \text{s.t.}	& 	& -y_2	& \geq	& 1\\
		&		&  y_1	&	& \geq	& 0\\
		&		&	&y_1,y_2& \geq	& 0\\
\multicolumn{6}{l}{\text{Infeasible, because $1 \leq -y_2 \leq 0$ contradictory}}
\end{array}
$