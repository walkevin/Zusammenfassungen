\section{Vektoranalysis}
\subsection{Differentialoperatoren}
\begin{tabular}{ll}
$\grad f := \nabla f = \begin{pmatrix} \partial_1 f\\ \partial_2 f\\ \partial_3 f\end{pmatrix}$	& Richtung und Betrag des gr�ssten Anstiegs von $f$\\
$\rot K := \nabla\times K = \begin{pmatrix} \partial_2 K_3 - \partial_3 K_2\\ \partial_3 K_1 - \partial_1 K_3 \\ \partial_1 K_2 - \partial_2 K_1\end{pmatrix}$	& lokale Zirkulationsrate von $K$\\
$\dgz K := \nabla\cdot K = \partial_1 K_1 + \partial_2 K_2 + \partial_3 K_3$	& lokale Produktionsrate von $K$\\
$\bigtriangleup f:= \dgz\grad f = \partial_1^2 f + \partial_2^2 f + \partial_3^2 f$	& lokale Diffusionsrate von $f$
\end{tabular}
\subsection{Lokale Eigenschaften}
$
 \rot\grad f = 0 \qquad \dgz\rot K = 0\\
 \rot K = 0 \Leftrightarrow \text{lokal $\exists f$ mit $\grad f = K$}\\
 \dgz K = 0 \Leftrightarrow \text{lokal $\exists L$ mit $\rot L = K$}
$

\subsection*{Globale Eigenschaften}
Folgende Eigenschaften eines Vektorfelds $K$ sind �quivalent:
\begin{itemize}
 \item $K$ besitzt ein Potential, d.h., ein Skalarfeld $f$ mit $\grad f = K$\\
 \item Das Integral von $K$ �ber jeden Weg h�ngt nur von Anfangs- und Endpunkt ab.\\
 \item Das Integral �ber jeden geschlossenen Weg ist Null.\\
 \item $K$ heisst konservativ.
\end{itemize}

\subsection*{Vektorielles Linienintegral}
$\varphi:[a,b]\rightarrow C\subset\mathbb{R}^3$
\begin{align*}
  \text{Formel:}\quad&\int_C K(x)\cdot dx := \int_a^b K(\varphi(x))\cdot |\varphi'(t)| dt\\
  \text{Satz von Stokes:}\quad & \int_C K(x)\cdot dx = \int_F \langle\rot K, n\rangle dF\quad F = \text{Fl�che in}\ \mathbb{R}^3, C = \partial F, \text{Kurve}\\
  \text{Satz von Green:}\quad & \int_{\partial B} \left\langle K\tbinom{x}{y},d\tbinom{x}{y}\right\rangle = \int_B (Q_x-P_y) dx dy
\quad K = \begin{pmatrix}P\\Q\\0\end{pmatrix}, n = \begin{pmatrix}0\\0\\1\end{pmatrix}\\
\end{align*}
$n =$ Einheitsnormalenvektor\\
Die Kurve heisst Feldlinie von $K$, falls gilt $\varphi'(t) = K(\varphi(t))$\\
Zur Fl�chenberechnung im $\mathbb{R}^2$ w�hlt man $K\binom{x}{y} = \binom{0}{x}, \binom{-y}{0},\binom{-y/2}{x/2} \Rightarrow\rot K = 1$\\
Interpretation: Arbeit oder Zirkulation von $K$ entlang $C$

\subsection*{Vektorielles Fl�chenintegral}
\begin{align*}
  \text{Formel $\mathbb{R}^2$:}\quad& \int_{\gamma} \langle K,n \rangle |dx| = \int_a^b K(\gamma(t))\cdot n(\gamma(t))\cdot |\gamma^(t)| dt\\
  \text{Formel $\mathbb{R}^3$:}\quad& \int_{\partial B} \langle K,n \rangle d\partial B := \int_D \left\langle K\left( \varphi\binom{u}{v}\right), (\varphi_u\times\varphi_v) \right\rangle du dv\\
  \text{Satz von Gauss:}\quad& \int_{\partial B} \langle K,n \rangle = \int_B \div K d\mu(\vec(x))\\
\end{align*}

Die Randkurve $\gamma$ wird so orientiert, dass die Fl�che stets links liegt, wenn man entlang $\gamma$ l�uft.\\
Der Einheitsnormalenvektor $n$ ist gleich $\frac{\varphi_u\times\varphi_v}{|\varphi_u\times\varphi_v|}$\\
Die Richtung von $n$ ist immer nach rechts von $\gamma$ aus gesehen.
Die Oberfl�che eines K�rpers im $\mathbb{R}^3$ wird stets nach aussen orientiert.