\section{Allgemeine Begriffe}
  \subsection{Funktionen}
  Eine Funktion $f:X\rightarrow Y$ heisst:
  \begin{align*}
   \text{injektiv}\quad & \forall x,x' \in X: f(x) = f(x') \Rightarrow x=x'\\
   \text{surjektiv}\quad & \forall y \in Y \ \exists\ x\in X:f(x) = y\\
   \text{bijektiv}\quad & \text{injektiv UND surjektiv}\\
    &\Rightarrow \forall y\in Y \ \exists !\ x\in X:f(x) = y\\\\
   \text{gerade}\quad & \forall x\in X: f(-x) = f(x)\qquad\text{//Achsensymmetrie Bsp: $\cos(x)$}\\
   \text{ungerade}\quad & \forall x\in X: f(-x) = -f(-x)\qquad\text{//Punktsymmetrie Bsp: $\sin(x)$}
\end{align*}
  \begin{tabular}{lll}
   Definitionsbereich von $f\quad$ & $dom(f)\quad$ &//$X$\\
   Wertebereich von $f\quad$ & $range(f)\quad$ & //$Y$\\
   Bildmenge von $f\quad$ & $image(f)\quad$ & //$\lbrace f(x) | x\in X\rbrace\subset Y$
  \end{tabular}
  \subsection{Stetigkeit}
  Eine Funktion $f:X\subset\mathbb{R}^m\rightarrow Y\subset{R}^n$ heisst stetig in $x_0\in X$, falls\\
  \begin{itemize}
   \item  $\forall\epsilon > 0 \exists \delta > 0 : \thinspace \forall x\in X: |x-x_0| < \delta \Rightarrow f(x)-f(x_0) < \epsilon$\\
   \item  f�r jede Folge $(x_k)$ in X mit $\lim{k\to\infty} x_k = x_0$ gilt: $\lim{k\to\infty} f(x_k) = f(x_0)$\\
   \item  $f$ ist stetig in $x_0 \Leftrightarrow f(x) = f(x_0)+o(1)$ f�r $x\to x_0$\\
  \end{itemize}
  $f$ ist stetig, falls $f$ stetig $\forall x_0\in X$ ist.\\
  
  \subsubsection*{Grundeigenschaften}
  \begin{enumerate}
   \item Die Zusammensetzung von stetigen Funktionen ist wieder stetig.
   \item Eine vektorwertige Funktion ist stetig gdw. alle Eintr�ge stetige Funktionen sind.
   \item Die Grundrechenarten sind stetig.
   \item Jede rationale Funktion ist stetig (wo definiert).
   \item Die Umkehrfunktion einer bijektiven und stetigen Funktion ist wieder stetig.
   \item Unstetige Funktionen sind z.B. $sgn(x)$ oder $\lfloor x\rfloor$
  \end{enumerate}
  
  \subsection{Mengen}
  Zu $x_0\in\mathbb{R}$ und $r > 0$ sei $B_r(x_0):=\lbrace x\in\mathbb{R}^n\thinspace |\thinspace |x-x_0| < r\rbrace$ der offene Ball mit Radius r um $x_0$\\
  Sei $X\subset\mathbb{R}^n$. Dann ist:
  \begin{enumerate}
   \item $X^0 := \lbrace x_0\in X | \ \exists \ r>0: B_r(x_0)\subset X\rbrace$ das Innere von X\\
   \item $\overline{X} := \lbrace y\in\mathbb{R}^n \ | \ \forall r > 0: B_r(y) \cap X \neq \emptyset \rbrace$ der Abschluss von X\\
   \item $\partial X := \overline{X} \backslash X^0$ der Rand von X\\
  \end{enumerate}
  Weiter heisst $X$:
  \begin{enumerate}
   \item offen, falls $X=X^0$\\
   \item abgeschlossen, falls $X=\overline{X}$\\
   \item Eine Teilmenge $Y\subset X$ heisst dicht in X, falls ihr Abschluss ganz $X$ ist.
  \end{enumerate}
\section{Grenzwerte}
  $f$ hat bei $x_0$ den Grenzwert $y_0\in\mathbb{R}^n$, falls gilt:\\
    $\forall\epsilon >0 \ \exists \ \delta>0: \ \forall x\in X: \ 0 < |x-x_0| < \delta\Rightarrow |f(x)-y_0| < \epsilon$\\
  F�r ausf�hrlichere Varianten siehe \url{http://www.math.ethz.ch/education/bachelor/lectures/hs2010/other/analysis1_itet/Grenzwerte.pdf}
  \subsection{Einige Grenzwerte von Funktionen}
  \begin{tabular}{lll}
   $\lim_{x\to\infty}\frac{1}{x} = 0$ 		& $\lim_{x\to 0+}\frac{1}{x} = \infty$ 	& $\lim_{x\to 0-}\frac{1}{x} = -\infty$\\
   $\lim_{x\to 0}\frac{\sin x}{x} = 1$   	& $\lim_{x\to 0}\frac{\ln(1+x)}{x} = 1$	& $\lim_{x\to 0}\frac{a^x-1}{x} = \ln a$\\
   $\lim_{x\to\infty}\frac{e^x}{x^q} = \infty$	& $\lim_{x\to\infty}\frac{x^q}{e^x} = 0$& $\lim_{x\to\infty} \ln x = \infty$\\
   $\lim_{x\to 0+}\ln x = \infty$		& $\lim_{x\to\infty}\frac{\ln x}{x^{\alpha}} = \lim_{x\to 0+} x^{\alpha}\cdot \ln x = 0$\\
   $\lim_{z\to 0}\frac{e^z-1}{z} = 1, z\in\mathbb{C}$ & $\lim_{x\to 0} \frac{\ln(1+x)}{x} = 1$	& $\lim_{n\to\infty} \left(1+\tfrac{x}{n}\right)^n = \exp(x)$
  \end{tabular}
  \subsection{S�tze �ber Grenzwerte von Funktionen}
  \begin{tabular}{ll}
   $\lim_{x\to x_0}[f(x)+g(x)] = \lim_{x\to x_0}f(x) + \lim_{x\to x_0}g(x)$ & $\lim_{x\to x_0} cf(x) = c \lim_{x\to x_0}f(x)$\\
   $\lim_{x\to x_0}[f(x)g(x)] = \left[\lim_{x\to x_0}f(x)\right] \left[\lim_{x\to x_0}g(x)\right]$ & $\lim_{x\to x_0}\frac{f(x)}{g(x)} = \frac{\lim_{x\to x_0}f(x)}{\lim_{x\to x_0}g(x)}$
  \end{tabular}\\
  Verkettung: $\lim_{x\to x_0}f(x) = u_0$ und $g(u)$ stetig in $u_0\Rightarrow\lim_{x\to x_0}g[f(x)] = g(u_0)$
  \subsubsection*{Regel von Bernoulli-de l'H�pital}
  $\lim_{x\to x_0}\frac{f(x)}{g(x)} = \lim_{x\to x_0}\frac{f'(x)}{g'(x)}$, falls $\lim_{x\to x_0}f(x) = \lim_{x\to x_0}g(x) = (0\vee\infty)$\\
  \subsubsection*{Majoranten- u. Minorantenkriterium}
  \begin{itemize}
   \item Ist $\lim_{x\to x_0} g(x) = 0$ und $|f(x)|\leq|g(x)|$ f�r alle $x$ nahe $x_0$, so gilt $\lim_{x\to x_0} f(x) = 0$\\
   \item Ist $\lim_{x\to x_0} g(x) = \infty$ und $f(x) > g(x)$ f�r alle $x$ nahe $x_0$, so gilt $\lim_{x\to x_0} f(x) = \infty$
  \end{itemize}
  \subsubsection*{Sonstige Tricks}
  \begin{enumerate}
   \item Bei $\lim_{x\to x_0}\frac{P(x)}{Q(x)},P,Q$ Polynomfunktionen, durch das Polynom mit dem h�chsten Grad teilen und Term damit erweitern.\\
	 Beispiel: $\lim_{x\to \infty}\frac{2x^2+3x-1}{x^2+5x+7} = \lim_{x\to \infty}\frac{\frac{1}{x^2}}{\frac{1}{x^2}}\cdot \frac{2x^2+3x-1}{x^2+5x+7} =
		    \lim_{x\to \infty}\frac{2+\frac{3}{x}-\frac{1}{x^2}}{1+\frac{5}{x}+\frac{7}{x^2}} = 2$\\
   \item $a^x = e^{\ln(a)\cdot x}$\\
  \end{enumerate}
  \subsubsection*{Beispiel}
  $\lim_{x\to\infty}\left(\frac{x}{x+1}\right)^x\quad   ``='' 1^{\infty}\quad\rightarrow\text{indefinit, falscher Ansatz!}\\
   =\lim_{x\to\infty}\exp\left(\ln\left(\frac{x}{x+1}\right)\cdot x\right)\quad\text{Trick 2}\\
   \text{Da $e^x$ �berall stetig, ist die Formel der Verkettung erlaubt}\\
   =\lim_{x\to\infty}\exp\left(\ln\left(\frac{x}{x+1}\right)\cdot x\right) = \exp\left(\lim_{x\to\infty}\left(\ln\left(\frac{x}{x+1}\right)\cdot x\right)\right)\quad\text{Verkettung}\\
   \text{ }\qquad\lim_{x\to\infty}\left(\ln\left(\frac{x}{x+1}\right)\cdot x\right)\\
   \text{ }\qquad = \lim_{x\to\infty} \frac{ \ln \left(\frac{x}{x+1}\right) }{\frac{1}{x}}\\
   \text{ }\qquad = \lim_{x\to\infty} \frac{ \frac{1}{(x+1)^2} \cdot \frac{1}{\frac{x}{x+1}} }{ -\frac{1}{x^2} }\quad\text{l'H�pital}\\
   \text{ }\qquad = \lim_{x\to\infty} \frac{1}{(x+1)^2} \cdot \frac{x+1}{x} \cdot -x^2 = \lim_{x\to\infty} -\frac{x^2}{x^2+x}\\
   \text{ }\qquad = \lim_{x\to\infty} -\frac{ \frac{1}{x^2} }{ \frac{1}{x^2} } \cdot \frac{x^2}{x^2+x}\quad\text{Trick 1}\\
   \text{ }\qquad = \lim_{x\to\infty} -\frac{1}{1+\frac{1}{x}} = -1\\
   \lim_{x\to\infty}\left(\frac{x}{x+1}\right)^x = e^{-1} = \frac{1}{e}
$
  \subsection{Asymptoten}
  \begin{enumerate}
   \item $f,g$ sind zueinander asymptotisch, falls $\lim_{x\to \infty} f(x)-g(x) = 0$\\
   \item Der Graph von g(x) heisst Asymptote zu f, falls g(x) eine lineare Funktion $px+q$ ist.
  \end{enumerate}
  \begin{tabbing}
  Beispiel: \=$f(t) = \frac{t^2-t-5}{t-3} = t+2+\frac{1}{t-3}$\\
	    \>$\lim_{x\to \infty}f(t)-g(t) = \lim_{x\to \infty}t+2+\frac{1}{t-3}-g(t)=0$\\
	    \>$\Rightarrow g(t) = t+2$\\   
  \end{tabbing}


% \end{multicols}
\section{Folgen und Reihen}
% \begin{multicols}{2}
\subsection{Allgemeines �ber Reihen}
$
\begin{array}{lll}
  \sum a_k	& \text{konvergent}		& s = \sum_{k=1}^{\infty} a_k = \lim_{n\to\infty} s_n\\
  \sum a_k	& \text{absolut konvergent}	& \sum |a_k|\enspace\text{ist konvergent}\\
  \sum a_k \enspace\text{konvergent}	&	& \Rightarrow (a_k) \enspace\text{ist eine Nullfolge}
\end{array}
$
\subsection{Grenzwerte spezieller Zahlenfolgen}
Die Grenzwerts�tze f�r Funktionen sind auch f�r Zahlenfolgen anwendbar.\\
$
\begin{array}{ll}
  \lim_{n\to\infty} \left(1+\frac{1}{n}\right)^n = e = 2.7182818284... 	& \lim_{n\to\infty} \left(1+\frac{x}{n}\right)^n = e^x\\
  \lim_{n\to\infty} \sqrt[n]{a} = 1,\quad a > 0		       		& \lim_{n\to\infty} \sqrt[n]{n} = 1\\
  \lim_{n\to\infty} \frac{a^n}{n!} = 0					& \lim_{n\to\infty} a^n = 0,\quad |a| < 1\\
  \lim_{n\to\infty} n\left(\sqrt[n]{a} - 1\right) = \ln a		& \lim_{n\to\infty} \frac{\log n}{n} = 0
\end{array}
$
\subsection{Spezielle Reihen}
$
\begin{array}{ll}
  \sum_{k=0}^n k = \frac{n(n+1)}{2}\quad\text{Gauss'sche Summenformel}			& \sum_{k=0}^n k^2 = \frac{n(n+1)(2n+1)}{6}\\
  \sum_{k=0}^n k^3 = \left(\frac{n(n+1)}{2}\right)^2					& \sum_{k=0}^n k^p = \frac{(n+1)^{p+1}}{p+1}\\
  \sum_{k=0}^n q^k = \frac{q^{n+1}-1}{q-1}\quad |q| < 1\quad\text{Geometrische Reihe}	& \sum_{k=0}^{\infty} q^k = \frac{1}{1-q}\\
  \sum_{k=1}^{\infty} \frac{1}{k} = \infty\quad\text{Harmonische Reihe}			& \sum_{k=1}^{n} \frac{1}{k} \approx \ln(n) + 0.557...\\
  \sum_{k=1}^{\infty} \frac{1}{k^s} \text{f�r $s>1$ konvergent$\quad$Riemann'sche Zetafunktion}\\
  \sum_{k=1}^{\infty} (-1)^{k-1}\frac{1}{k} = \ln 2\quad\text{alternierende harmonische Reihe}
\end{array}
$
\subsection{Umordnung von Reihen}
Jede absolut konvergente Reihe bleibt absolut konvergent nach beliebiger Umordnung. Ist die Reihe nicht absolut konvergent,
so existiert eine Umordnung, die divergiert.\\
\begin{align*}
\text{Beispiel:} & \underbrace{\sum_{k=1}^{\infty} (-1)^k}_{\text{nicht abs.konv.}} = 0
  \neq \underbrace{\sum_{k=1}^{\infty} 1^{2k}}_{=\infty} + \underbrace{\sum_{k=1}^{\infty} 1^{2k+1}}_{= -\infty}
\end{align*}
\subsection{Rechenregeln}
\begin{math}
  \sum_{k = k_0}^{\infty} a_k = \sum_{k = k_0}^{k_1-1} a_k + \sum_{k = k_1}^{\infty} a_k\\
  \sum_{k = k_0}^{\infty} (a_k+b_k) = \sum_{k = k_0}^{\infty} a_k + \sum_{k = k_0}^{\infty} b_k\\
  \sum_{k = k_0}^{\infty} c\cdot a_k = c\cdot \sum_{k = k_0}^{\infty} a_k\\
  a_k \leq b_k \forall k\Rightarrow \sum_{k=k_0}^{\infty} a_k \leq \sum_{k = k_0}^{\infty} b_k\\
  \sum_{l = l_0}^{\infty} \sum_{k = k_0}^{\infty} |a_{k,l}| < \infty \Rightarrow \text{absolut konvergent}
\Rightarrow\sum_{l = l_0}^{\infty} \sum_{k = k_0}^{\infty} a_{k,l} = \sum_{k = k_0}^{\infty} \sum_{l = l_0}^{\infty} a_{k,l}\\
  \left(\sum_{k = k_0}^{\infty} a_k\right)\cdot \left(\sum_{l = l_0}^{\infty} b_l\right) = \sum_{k = k_0}^{\infty} \sum_{l = l_0}^{\infty} a_kb_l
\end{math}
\subsection{Konvergenzkriterien}

\begin{tabular}{ll}
  Kriterium von Leibniz\\
  $(a_k)$ alternierend und $(|a_k|)$ monotone Nullfolge			& $\Rightarrow\sum a_k$ ist konvergent\\
  Majorantenkriterium\\
  $(|a_k|) \leq c_k$ f�r fast alle $k$, $\sum c_k$ konvergent		& $\Rightarrow\sum a_k$ ist absolut konvergent\\
  Quotientenkriterium\\
  $\left| \frac{a_{k+1}}{a_k} \right| \leq q$ mit $0<q<1$ f�r fast alle k	& $\Rightarrow\sum a_k$ ist absolut konvergent\\
  Wurzelkriterium\\
  $\sqrt[k]{|a_k|} \leq q$ mit $0<q<1$ f�r fast alle k				& $\Rightarrow\sum a_k$ ist absolut konvergent\\
\end{tabular}\\
Bemerkungen: Ergibt das Quotientenkriterium genau 1, so sagt es nichts �ber die Konvergenz aus und ist nicht anwendbar.\\

\section{Potenzreihen}
Ein Ausdruck der Form $f(z) = \sum_{k=0}^{\infty} a_kz^k$ heisst Potenzreihe
  \subsection{Konvergenzradius bei Potenzreihen}
  \begin{tabular}{ll}
    Quotientenkriterium	& $r = \lim_{k\to\infty} \left|\frac{a_k}{a_{k+1}}\right|$\\
    Wurzelkriterium	& $r = \frac{1}{\limsup_{k\to\infty}\left(\sqrt[k]{|a_k|}\right)}$\\
    geometrische Reihe	& $r = 1$ bei $\sum_{k=0}^{\infty} q^k$ falls $|q| < 1$\\
    binomische Reihe	& $r = 1$ bei $\sum_{k=0}^{\infty} \binom{\alpha}{k}\cdot z^k$ falls $|z| < 1$
  \end{tabular}\\
  \begin{align*}
    \text{Beispiel:}\quad	& \text{Konv.radius von}\enspace f(z) = \frac{1}{z-2}\\
			  & \frac{1}{z-2} = -\frac{1}{2-z} = -\frac{1}{2(1-\tfrac{z}{2})}\\
			  & -\frac{1}{2}\cdot\frac{1}{1-\tfrac{z}{2}} = -\frac{1}{2}\sum_{k=0}^{\infty}\left(\frac{z}{2}\right)^k\\
			  & \left|\frac{z}{2}\right| < 1 \Leftrightarrow |z| < 2 = \text{Konvergenzradius}
  \end{align*}
  \subsection{Binomische Reihe}
  $\sum_{k=0}^{\infty} \binom{\alpha}{k} \cdot z^k = (1+z)^{\alpha}$ f�r $|z| < 1$\\
  $\begin{array}{ll}
  \alpha = n\in\mathbb{Z}^{\geq 0}:	& (1+x)^n = \sum_{k=0}^{\infty} \binom{n}{k}\cdot z^k\\
  \alpha = -n; n\in\mathbb{Z}^{\geq 0}:& \frac{1}{(1-z)^n} = \sum_{k=0}^{\infty} \binom{n+k-1}{k}\cdot z^k\\
  \alpha = -1				& \frac{1}{1-z} = \sum_{k=0}^{\infty} z^k\\
  \alpha = \frac{1}{2}			&\sqrt{1+z} = \sum_{k=0}^{\infty} \binom{1/2}{k}\cdot z^k
  \end{array}
$
  \subsection{Potenzreihenentwicklung}
  Die Potenzreihenentwicklung kann man bei Funktionen brauchen, die irgendwas mit $e^x$ oder sonstigen Funktionen, deren Potenzreihenentwicklung bekannt ist, verwendet werden.\\
  \begin{align*}
   \text{Beispiel:}\quad &\ln(1+y) = x \Leftrightarrow 1+y = e^x = 1+x+\frac{x^2}{2}+\frac{x^3}{6}+\dotsc\\
	&y = x + \frac{x^2}{2}+\dotsc\\
	&\text{Ansatz:}\enspace x = y+ay^2+by^3+\dotsc\\
	&y=(y+ay^2+by^3+\dotsc)+\frac{1}{2}(y+ay^2+by^3+\dotsc)^2+\frac{1}{6}(y+ay^2+by^3+\dotsc)^3+\dotsc\\
	&y=y + y^2(a+\tfrac{1}{2})+y^3(b+\tfrac{1}{2}\cdot 2a+\tfrac{1}{6}) + O(y^4)\\
	&\text{Alle Terme = 0 setzen liefert:} a = -1/2, b = 1/3\\
	&\ln(1+y) = y-\frac{y^2}{2}+\frac{y^3}{3}-\dotsc\pm\frac{y^k}{k}
  \end{align*}

\section{Differentialrechnung}
\subsection{klein o-Gross O-Notation}
Funktion f ist $O(g(x))$ falls $\left|\frac{f(x)}{g(x)}\right|$ beschr�nkt f�r $x\to x_0$\\
Funktion f ist $o(g(x))$ falls $\left|\frac{f(x)}{g(x)}\right| \xrightarrow{x\to x_0} 0$
\subsection{S�tze}
\begin{tabular}{lp{13cm}}
  Diffbarkeit		& $f$ ist diffbar in $x_0$ mit Ableitung $f'(x_0) \Leftrightarrow f(x) = f(x_0) + f'(x_0)\cdot(x-x_0) + o(x-x_0); x\to x_0$ \newline
			  $f$ ist diffbar, falls diffbar in jedem $x_0 \in X$\\\hline
  Zwischenwertsatz	& Sei $f:[a,b]\rightarrow\mathbb{R}$ stetig. Dann nimmt $f$ jeden Wert zwischen $f(a)$ und $f(b)$ an. 
D.h: $f(a) \leq f(b) \Rightarrow [f(a),f(b)]\subset image(f)$\\\hline
  Mittelwertsatz	& Sei $f:[a,b]$ stetig und auf $]a,b,[$ diffbar, so existiert eine Stelle t in $]a,b[$ mit $f'(t) = \frac{f(b)-f(a)}{b-a}$\\\hline
  Potenzreihenidentit�tssatz	& Stellen zwei Potenzreihen f�r $|x| < \epsilon$ dieselbe Funktion dar, so sind sie bereits gliedweise gleich.\\\hline
  Hauptsatz der Infinitesimalrechnung	& $\int_a^b f(x)dx = F(b)-F(a)$\\
\end{tabular}

\subsection{Supremum,Infimum}
\begin{tabular}{ll}
  Supremum, $\sup(X)$	& Kleinste obere Schranke f�r die Menge X\\
  Infimum, $\inf(X)$	& Gr�sste obere Schranke f�r die Menge X
\end{tabular}\\
\subsubsection*{Rechenregeln}
$
\begin{array}{ll}
  \sup(X+b) = \sup(X) + b		& \sup(X+Y) = \sup(X)+\sup(Y); X,Y\neq\emptyset\\
  \sup(c\cdot X) = c\cdot\sup(X); c>0	& \sup(c\cdot X) = c\cdot\inf(X); c<0\\
  \inf(X) = -\sup(-X)
\end{array}
$
\subsection{Taylorapproximation}
$j_{x_0}^n f(x) = \sum_{k=0}^n \frac{f^{(k)}(x_0)}{k!}(x-x_0)^k$ heisst das n-te Taylor-Polynom von $f$ an der Stelle $x_0$.\\
$R_n(x) = \frac{f^{(n+1)}(t)}{(n+1)!}(x-x_0)^{n+1}$ ist dabei das n-te Restglied.\\
Mit diesem Term l�sst sich der Fehler absch�tzen, denn es gilt: $f(x) = j_{x_0}^n f(x) + R_n(x)$
\subsection{Newtonverfahren zur Nullstellenfindung}
$x_{n+1} := x_n-\frac{f(x_n)}{f'(x_n)}\qquad \lim_{n\to\infty} x_n = \text{Nullstelle}$
\section{Integralrechnung}
  Partielle Integration $\int_a^b f'(x)g(x)dx = \left[ f(x)g(x)\right]_a^b - \int_a^b f(x)g'(x) dx$\\
\subsection{Standardsubstitutionen}
  \begin{tabular}{lll}
  \hline
  $\int f(x, \sqrt{ax+b})dx$		& $t = \sqrt{ax+b}$	& $dx = \frac{2t dt}{a}$\\\hline
  $\int f(g'(x),g(x))dx$		& $t = g(x)$		& $dx = \frac{dt}{g'(x)}$\\\hline
  $\int f(x,\sqrt{a^2-x^2})dx$		& $x = a\cdot\sin t$	& $dx = a\cdot\cos(t)dt$\\\hline
  $\int f(x,\sqrt{x^2+a^2})dx$		& $x = a\cdot\sinh t$	& $dx = a\cdot\cosh(t)dt$\\\hline
  $\int f(x,\sqrt{x^2-a^2})dx$		& $x = a\cdot\cosh t$	& $dx = a\cdot\sinh(t)dt$\\\hline
  $\int f(x, \sqrt{ax^2+bx+c})dx$	& \multicolumn{2}{l}{durch quad.Erg�nzung auf obere Formen bringen}\\\hline
  $\int f(e^x,\sinh x,\cosh x) dx$	& $t = e^x$		& $dx = \frac{dt}{t}$\\\hline
  $\int f(\sin x,\cos x)dx$		& $t = \tan\frac{x}{2}$	& $dx = \frac{2 dt}{1+t^2}$\\\hline
  \end{tabular}
\subsection{L�sungen einiger Integrale}
$
  \begin{array}{ll}
   \int \frac{dx}{ax+b} = \frac{1}{a}\ln|ax+b|+c	& \int (ax+b)^n dx = \frac{1}{a}\cdot\frac{1}{n+1}\cdot (ax+b)^{n+1}\\
   \int \sin^2(ax+b) dx = \frac{x}{2}-\frac{1}{4a}\cdot\sin(2as+2b)+c	& \int \cos^2(ax+b) dx = \frac{x}{2}+\frac{1}{4a}\cdot\sin(2as+2b)+c\\
   \int \frac{f'(x)}{f(x)} dx = \ln |f(x)|+c		& \int [f'(x)]^n\cdot f(x) dx = \frac{1}{n+1}\cdot[f(x)]^{n+1}+c\\
  \end{array}
$

\subsection{Konvergenzkriterien f�r uneigentliches Integral}
\begin{itemize}
 \item F�r $f$ stetig auf $[a,\infty[$\\
	Majorentenkriterium: $\int_a^{\infty}f(x) dx$ konvergiert, falls $|f(x)| \leq \frac{c}{x^s}, s>1$\\
	Minorantenkriterium: $\int_a^{\infty}f(x) dx$ divergiert, falls $f(x) \geq \frac{c}{x^s}, s\leq 1$\\
 \item F�r $f$ stetig auf $]a,b]$, auf $[a,b[$ integrierbar.\\
	Majorentenkriterium: $\int_a^b f(x) dx$ konvergiert, falls $|f(x)| \leq \frac{c}{(x-a)^s}, s<1$\\
	Minorantenkriterium: $\int_a^b f(x) dx$ divergiert, falls $f(x) \geq \frac{c}{(x-a)^s}, s\geq 1$ und $c>0$\\
\end{itemize}