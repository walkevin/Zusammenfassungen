\section{Vektorwertige Funktionen}
  \subsection{Funktionalmatrix (Jacobimatrix)}
  Sei $U\subset\mathbb{R}^m$ offen; $f=\begin{pmatrix}f_1\\\vdots\\f_m\end{pmatrix}: U\rightarrow\mathbb{R}^m$\\
  Dann ist $\nabla f = \begin{pmatrix}\frac{\partial f_1}{\partial x_1} & \dotsb & \frac{\partial f_1}{\partial x_n}\\
				      \vdots				& \ddots & \vdots\\
				      \frac{\partial f_m}{\partial x_1} & \dotsb & \frac{\partial f_m}{\partial x_n}
                       \end{pmatrix}$\\
  eine $m\times n$-Matrix, genannt die Funktionalmatrix resp. Jacobimatrix\\
  \subsubsection*{Zusammenhang mit Hessematrix}
  Sei $P:U\rightarrow\mathbb{R}^2$ eine $C^2$-Funktion.\\
  Dann ist $(\nabla P)^T$ eine vektorwertige Funktion mit 2 Variablen\\
  $\Rightarrow$ Die Funktionalmatrix von $(\nabla P)^T$ ist eine $n\times n$-Matrix, die genau der Hessematrix $\nabla^2 P$ entspricht.
  \subsection{Funktionaldeterminante (Jacobideterminante)}
  $\det\nabla f(x_0)$ heisst Funktionaldeterminante.
  \subsection{Kettenregel f�r vektorwertige Funktionen}
  \begin{math}
    \text{Formel:}\quad\nabla(g\circ f) = (\underbrace{\nabla g)\bigl(f(x)\bigr)}_{\text{m Spalten}}
      \cdot \underbrace{(\nabla f)\bigl(x\bigr)}_{\text{m Zeilen}}\quad\leftarrow\text{Matrixmultiplikation}\\
  \end{math}

  \subsection{Inverse}
  Sind $f,g$ zueinander invers, so ist $\nabla f$ eine invertierbare Matrix und d.h.\\
  $\nabla g\bigl(f(x)\bigr) = (\nabla f)^{-1}\Leftrightarrow g(y) = \biggl(\nabla f\bigl(g(y)\bigr)\biggr)^{-1}$