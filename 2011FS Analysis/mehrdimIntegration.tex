\section{Mehrdimensionale Integration}
\subsection{Grundeigenschaften}
Notation: $\mu(z)$ entspricht dem eindimensionalen dx,dy,... und bezeichnet das Differential.\\
$\mu(B)$ wiederum bezeichnet das Volumen von B.
\begin{enumerate}
  \item $\int_{A\cup B} f(z) \mu(z) = \int_{A} + \int_{B} - \int_{A\cap B}\quad\text{vgl. Inklusion-Exklusion}$\\
  \item $\int_{B} f(z)\mu(z) = 0\quad\text{,falls $\mu(B)$ = 0}$\\
  \item $\int_{B} 1 \mu(z) = \mu(B)\quad\text{Volumen}$\\
  \item $\int_{B} f(z) + g(z)\mu(z) = \int_{B} f(z)\mu(z) + \int_{B} g(z)\mu(z)\quad\text{Regel der Linearit�t}$\\
  \item $\int_{B} \lambda\cdot f(z)\mu(z) = \lambda\cdot \int_{B} f(z)\mu(z)\quad\text{Regel der Konstanten}$\\
  \item $f(z) \leq g(z) \forall z\in B\Rightarrow \int_{B} f(z)\mu(z) \leq \int_{B} g(z)\mu(z)$\\
  \item $\left|\int_{B} f(z)\mu(z)\right|\leq \int_{B} |f(z)|\mu(z)$\\
  \item $f(x) \leq g(x) \forall x\in B, f,g:B\rightarrow\mathbb{R} \enspace\text{stetig}\\
	\int_{B} g(x)-f(x)\mu(z) = \mu\left(\left\lbrace\left.\begin{pmatrix}x\\\vdots\\x_{n+1}\end{pmatrix}\right| x\in B, f(x) \leq x_{n+1} \leq g(x)\right\rbrace\right)
	\\\text{Berechnung des n+1 dimensionalen Volumens}$
\end{enumerate}
\subsection{Satz von Fubini}
$\int_{I\times J}f(x,y) d(x,y) = \int_{I}\int_{J}f(x,y) dx dy = \int_{J}\int_{I}f(x,y) dy dx$
\subsection{Substitution}
\begin{align*}
\text{Formel:}\quad &\int_{B}f(x)\mu(x) = \int_{\tilde{B}}f\bigl(\varphi(\tilde{x})\bigr)\cdot |\nabla\varphi(\tilde{x})| \cdot \mu(\tilde{x})\\
\text{Beispiel:}\quad &\text{Berechne die Fl�che von }B = \left\lbrace\left.\binom{x}{y}\right| x^2+y^2 \leq R^2\right\rbrace\\
&\text{Polarkoordinaten:}\enspace \binom{x}{y} = \varphi\binom{r}{\psi} = \binom{r\cos\psi}{r\sin\psi},\quad|\nabla\varphi| = r\\
&\tilde{B} = \left\lbrace\left.\binom{r}{\varphi}\right| \begin{matrix}0\leq r\leq R\\-\pi\leq\psi\leq\pi\end{matrix}\right\rbrace\\
&\int_B 1\cdot\mu\binom{x}{y} = \int_0^R\int_{-\pi}^{\pi} r \ d\varphi \ dr = \left[\frac{r^2}{2}\right]_{r=0}^{r=R}\cdot 2\pi = R^2\pi
\end{align*}
\subsection{Masse}
\begin{align*}
  \text{Diskret:}\quad & M = \sum m_i\\
  \text{Kontinuierlich:}\quad & M = \int_B \rho(x)\mu(x)\\
  &\text{``Dichte $\cdot$ Volumen''}
\end{align*}
\subsection{Schwerpunkt}
\begin{align*}
  \text{Diskret:}\quad & S = \frac{\sum m_i x_i}{\sum m_i}\\
  \text{Kontinuierlich:}\quad & S = \frac{\int_B \rho(x)\cdot\vec{x}\mu(x)}{\int_B \rho(x)\mu(x)}\\
\end{align*}
\subsection{Tr�gheitsmoment}
\begin{align*}
  \text{Formel:}\quad & J = \int_B \rho(\vec{r})\cdot |\vec{r_{\bot}}|^2 dB\\
  &\text{$|\vec{e}\times\vec{r}| = |\vec{r_{\bot}}|$ ist der Abstand vom Punkt $\vec{r}$ zur Drehachse $\vec{e}$}
\end{align*}
\subsection{Potentiale}
Ist $K(x) = (\nabla f)^T$ dann ist $f$ ein Potential von $K$ (Verallgemeinerung von Stammfunktion)\\
Bsp: $K$ Kraftfeld, $f$ zugeh�rige potentielle Energie\\
Ein $C^1$ Vektorfeld $K = (K_1 \dotsc K_n)^T$ hat lokal ein Potential gdw. $\forall i \forall j: \frac{\partial K_i}{\partial x_j} = \frac{\partial K_j}{\partial x_i}$\\
\subsubsection*{explizite Berechnung}
im $\mathbb{R}^2$ gilt: $(\nabla f)^T = K\binom{x}{y} = \binom{P}{Q}\quad\text{f�r}\ f:\mathbb{R}^2\rightarrow\mathbb{R}\\
\Rightarrow \text{2 DGL:}\qquad P\binom{x}{y} = \frac{\partial f}{\partial x}\quad Q\binom{x}{y} = \frac{\partial f}{\partial y}$\\
\begin{enumerate}
 \item Berechne $f_1\binom{x}{y} = \int P\binom{x}{y}dx\\
       \Rightarrow \text{Allg.L�sung der 1.DGL:}\enspace f = f_1+g(y)\quad\text{$g(y)$ muss von $x$ unabh�ngig sein!}$\\
 \item $f$ in 2.DGL einsetzen: $Q = \frac{\partial f_1}{\partial y} + g'(y) \Leftrightarrow g = \int Q-\frac{\partial f_1}{\partial y}dy\\
\Rightarrow$ Falls $Q-\frac{\partial f_1}{\partial y}$ nicht unabh�ngig von x ist, existiert kein Potential!
\end{enumerate}
\begin{align*}
  \text{Beispiel:}\quad &K\binom{x}{y} = \binom{x^3+xy^2}{x^2y-y^5}\\
  \text{1.}\enspace& f_1\binom{x}{y} = \int x^3+xy^2 dx = \frac{x^4}{4} + \frac{x^2y^2}{2}\\
  &\Rightarrow \text{Allg.L�sung der 1.DGL:}\enspace f = f_1+g(y)\\
  \text{2.}\enspace& x^2y-y^5 = x^2y + g'(y) \Leftrightarrow g = \int y^5 dy = \frac{y^6}{6}+c\\
  &\Rightarrow f = \frac{x^4}{4} + \frac{x^2y^2}{2} + \frac{y^6}{6}+c
\end{align*}
\subsection{skalares Linienintegral}
$\varphi:[a,b]\rightarrow C\subset\mathbb{R}^n$
\begin{align*}
  \text{Formel:}\quad&\int_C f(x)|d\vec{x}| := \int_a^b f(\varphi(x))\cdot |\varphi'(t)| dt\quad f:C\rightarrow\mathbb{R}\\
  \text{Hauptsatz:}\quad&\int_C \grad f(x) \cdot dx = f(q)-f(p),\quad\text{C ist der Weg von p nach q}\\
  \text{L�nge einer Kurve:}\quad&\int_C 1|dx| = \int_a^b |\varphi'(t)| dt\\
  \text{L�nge eines Graph:}\quad&\int_a^b\sqrt{1+\psi'(t)^2}dt\quad\varphi:[a,b]\rightarrow\mathbb{R}^2,t\mapsto\binom{t}{\psi(t)}
\end{align*}
\subsection{skalares Fl�chenintegral}
$\delta$ bezeichnet die Fl�che (wie $\mu$ das Volumen)\\
$\mathbb{R}^2 \supset B \xrightarrow{\varphi} F\subset\mathbb{R}^3$
\begin{align*}
   \text{Formel:}\quad&\int_F f(x)\delta(x) := \int_B f\left(\varphi\binom{u}{v}\right)\cdot
     \left|\frac{\partial\varphi}{\partial u}\times\frac{\partial\varphi}{\partial v}\right|\ du\thinspace dv\\
   \text{Fl�che:}\quad&\int_C 1\delta x = \int_B \left|\frac{\partial\varphi}{\partial u}\times\frac{\partial\varphi}{\partial v}\right| du\thinspace dv\\
   \text{Fl�che eines Graphs:}\quad&\int_B \sqrt{1+\psi_u^2+\psi_v^2} du\thinspace dv\quad\varphi\binom{u}{v} = \begin{pmatrix}u\\v\\\psi(u,v)\end{pmatrix}
\end{align*}
\subsection{Formeln f�r Rotationsk�rper}
\begin{align*}
  \text{Masse o.�.:}\quad& \int_a^b\int_0^{r(z)}f\binom{\rho}{z}2\pi\rho\ d\rho\thinspace dz\\
  \text{Volumen:}\quad& \int_a^b \pi\cdot r(z)^2 dz\\
  \text{Fl�che:}\quad& \int_a^b 2\pi\cdot r(z)\sqrt{1+r'(z)^2} dz
\end{align*}