\section{Schr�dingergleichung}
\subsection{Allgemeines}
\begin{tabular}{p{4cm} >{$}p{15cm}<{$}}
``Herleitung'' der SG			& \begin{array}[t]{rl}
						& \Psi(x,t) = \Psi_0 \exp \left( \frac{i}{\hbar} (p_x x-Et) \right) \quad \text{(de Broglie)}\\
				\Rightarrow	& p_x \Psi(x,t) = -i\hbar \frac{\partial}{\partial x} \Psi(x,t)\\
				\Rightarrow	& E \Psi(x,t) = i\hbar \frac{\partial}{\partial t} \Psi(x,t)\\
						& \text{Via Korrespondenzprinzip gilt: $E = \frac{\hat{p}^2}{2m}$}\\
				\Rightarrow	& E \Psi(x,t) = \underbrace{i\hbar \frac{\partial}{\partial t}}_{E} \Psi(x,t) = \underbrace{\frac{-\hbar^2}{2m} \frac{\partial^2}{\partial x^2}}_{H} \Psi(x,t) = \frac{\hat{p}^2}{2m}\Psi(x,t)\\
						& \text{L�sungsansatz (Separation der Variablen): $\Psi(x,t) = \psi(x) \cdot \chi(t)$}\\
				\Leftrightarrow	& \psi(x) i\hbar \frac{\partial \chi(t)}{\partial t} = \chi(t) \cdot \frac{-\hbar^2}{2m} \frac{\partial^2}{\partial x^2} \psi(x)\\
				\Leftrightarrow	& \frac{i \hbar \dot{\chi(t)}}{\chi(t)} = \frac{-\hbar^2}{2m} \frac{\psi''(x)}{\psi(x)} = E\\
				\Rightarrow	& \text{ODE 1: } i\hbar \dot{\chi(t)} = E \chi(t) \Rightarrow \chi(t) = \exp(\frac{-i}{\hbar} Et)\\
						& \text{ODE 2: } \frac{-\hbar^2}{2m} \psi''(x) = E \psi(x)
                 			  \end{array}\\
allgemeine SG				& \hat{H} \Psi = \hat{E} \Psi\\
allg. zeitabh�ngige L�sung		& \psi(\mathbf{r},t) = \psi(\mathbf{r}) e^{-i\frac{E}{\hbar}t} = \psi(\mathbf{r})e^{-i\omega t}\\
Normierungsbedingung			& \int_{-\infty}^{\infty} \Psi^{\ast}(x) \Psi(x) dx = 1 = \int_{-\infty}^{\infty} |\Psi(x)|^2 dx\quad \text{falls $\Psi$ reell}\\
Aufenthalts wahrscheinlichkeitsdichte	& P(x) = \Psi^{\ast}(x)\Psi(x)\\
Delokalisation				& \text{vollst�ndige Delokalisation eines Teilchens, falls $P(x) = konst.$}\\
Heisenberg'sche Unsch�rferelation	& \Delta x \Delta p \geq \tfrac{1}{2} \hbar\\
Kommutator				& \left[\hat{A},\hat{B}\right] \equiv \hat{A}\cdot \hat{B} - \hat{B} \cdot \hat{A}\quad \text{Falls} \left[\hat{A},\hat{B}\right] = 0\Rightarrow \text{A und B gleichzeitig messbar}\\
Operatoren				& \begin{array}[t]{lll}
          				   \hat{r}	& \text{Orts-Operator}	& \hat{r} = (x,y,z)^T \\
					   \hat{p}	& \text{Impuls-Operator}	& \hat{p} = -i\hbar\nabla = -i\hbar (\partial / \partial x, \partial / \partial y , \partial / \partial z)^T\\
					   \hat{H}	& \text{Hamilton-Operator*}	& \hat{H} = \frac{\hat{p}^2}{2m} + E_p(\hat{r}) = -\frac{\hbar^2}{2m}\nabla^2 + E_p(\hat{r})\\
					   \multicolumn{3}{l}{\text{* f�r ein freies Teilchen im Potentialfeld}}
          				  \end{array}\\
\end{tabular}
\subsection{allgemeine L�sung f�r SG mit Potential}
\begin{tabular}{p{4cm} >{$}p{15cm}<{$}}
zeitunabh�ngige Schr�dingergleichung (1-dim)	& -\frac{\hbar^2}{2m}\frac{d^2\psi(x)}{dx^2} + E_{pot}(x)\psi(x) = E\psi(x)\\
L�sung der zeitunabh�ngigen SG		& \begin{array}[t]{ll}
						& \underbrace{\frac{-\hbar^2}{2m}}_{=: -C} \frac{\partial^2}{\partial x^2} \psi(x) = (E-E_{pot}) \psi(x)\\
						& \text{Ansatz: $\psi(x) = e^{ikx}$}\\
				\Rightarrow	&  Ck^2\cdot e^{ikx} = (E-E_{pot}) \cdot e^{ikx}\\
				\Leftrightarrow	& k = \pm\sqrt{\frac{E-E_{pot}}{C}} = \pm\sqrt{\frac{(E-E_{pot})\cdot 2m}{\hbar^2}}\\
				\Rightarrow	& \psi_k(x) = A_k e^{ikx} + B_k e^{-ikx}
                                      	  \end{array}\\
L�sung f�r $E_{pot} = 0$		& \psi_1(x) = Ae^{ikx} + Be^{-ikx}\quad k^2 = \frac{2mE}{\hbar^2}\\
L�sung f�r $E < E_{pot}$		& \psi_2(x) = Ce^{-\alpha x} + De^{\alpha x}\quad \alpha^2 = \frac{2m}{\hbar^2}(E_{pot}-E)\\
L�sung f�r $E > E_{pot}$		& \psi_3(x) = Ee^{i\alpha'x} + Fe^{-i\alpha' x}\quad \alpha'^2 = \frac{2m}{\hbar^2}(E-E_{pot})\\
Stetigkeitsbedingungen			& \psi_i = \psi_j \quad\text{und}\quad \frac{\partial \psi_i}{\partial x} = \frac{\partial \psi_j}{\partial x} \text{ f�r $x=0$}
\end{tabular}
\subsection{Spezifische Probleme}
\subsubsection{Potentialstufe}
\begin{tabular}{p{4cm} >{$}p{15cm}<{$}}
$E_{pot}$	& E_{pot} = \begin{cases} 0 & x < 0 \text{ (Bereich I)}\\ E_0 & x \geq 0 \text{ (Bereich II)}\end{cases}\\\midrule
allg. L�sung ($E < E_{pot}$)	& \begin{array}[t]{l}
      		  \psi_I(x) = Ae^{ikx} + Be^{-ikx}\quad k^2 = \frac{2mE}{\hbar^2}\\
		  \psi_{II}(x) = Ce^{-\alpha x} + De^{\alpha x}\quad \alpha^2 = \frac{2m}{\hbar^2}(E_{pot}-E)
      		  \end{array}\\
Stetigkeitsbedingungen	& \left.\psi_I\right|_{x=0} = \left.\psi_{II}\right|_{x=0} \quad\text{und}\quad \left.\frac{\partial \psi_I}{\partial x}\right|_{x=0} = \left.\frac{\partial \psi_{II}}{\partial x}\right|_{x=0}\\
Randbedingungen	& \lim_{x\to\infty} \psi_{II} < \infty \text{ (vgl. Normierungsbedingung)}\\
spez. L�sung	& \begin{array}[t]{ll}
            	  D = 0 \text{ (aus RB)}	&\\
		  B = \frac{(ik+\alpha)A}{ik-\alpha}	& C = \frac{2ikA}{ik-\alpha}
            	  \end{array}\\\midrule
allg. L�sung ($E > E_{pot}$)	& \begin{array}[t]{l}
      		  \psi_I(x) = Ae^{ikx} + Be^{-ikx}\quad k^2 = \frac{2mE}{\hbar^2}\\
		  \psi_{II}(x) = Ee^{i\alpha'x} + Fe^{-i\alpha' x}\quad \alpha'^2 = \frac{2m}{\hbar^2}(E-E_{pot})
      		  \end{array}\\
spez.L�sung	& \begin{array}[t]{ll}
            	  F = 0 \text{ (aus RB)}	&\\
		  B = \frac{(k-\alpha')A}{k+\alpha'}	& E = \frac{2kA}{k+\alpha'}
            	  \end{array}
\end{tabular}
\subsubsection{Potentialtopf (1D)}
\begin{tabular}{p{4cm} >{$}p{15cm}<{$}}
$E_{pot}$	& E_{pot} = \begin{cases} 0 & 0 < x < a \text{ Bereich I}\\ \infty & \text{ sonst (Bereich II)}\end{cases}\\
allg. L�sung	& \begin{array}[t]{l}
            	  \psi_I(x) = Ae^{ikx} + Be^{-ikx}\quad k^2 = \frac{2mE}{\hbar^2}\\
		  \Leftrightarrow A'\sin(kx) + B'\cos(kx)\\
		  \psi_{II}(x) = 0
            	  \end{array}\\
Randbedigungen	& \psi_I(x=0) = \psi_I(x=a) = 0\\
RB f�r periodische Probleme	& e^{ik\phi} = e^{ik(\phi+2\pi)} \Leftrightarrow 1 = e^{ik2\pi} \Leftrightarrow k = 0,\pm 1,\pm 2 ,... = m_l\\
spez. L�sung	& \psi_I(x) = A' \sin(kx)\quad k = \frac{n\pi}{a} \quad \Rightarrow \lambda = \frac{2\pi}{k} = \frac{2a}{n}\quad n = 1,2,3,...\\
Normierungsbedingung	& A'^2 \int_0^a \sin^2(kx) dx = A'^2 \frac{a}{2} = 1 \Leftrightarrow A = \sqrt{\frac{2}{a}}\\
Energieeigenwert	& E = \frac{\hbar^2 k^2}{2m} = \frac{n^2\pi^2\hbar^2}{2ma^2}
\end{tabular}
\subsubsection{Potentialtopf (3D)}
\begin{tabular}{p{4cm} >{$}p{15cm}<{$}}
SG in 3 Dimensionen			& \psi(x,y,z) = \psi_1(x)\psi_2(y)\psi_3(z)\quad\text{$\psi_{1,2,3}$ sind Wellenfkt. im 1-dim. Kasten}\\
Particle in a cube			& \begin{array}[t]{l}
                 			   \psi(x,y,z) = \sqrt{\frac{8}{a^3}} \sin(\tfrac{n_1\pi}{a} x) \sin(\tfrac{n_2\pi}{a} y) \sin(\tfrac{n_3\pi}{a} z)\\
					   E = \frac{\hbar^2 \pi^2}{2ma^2}(n_1^2+n_2^2+n_3^2)\\
                 			  \end{array}
\end{tabular}
\subsubsection{Der Tunneleffekt}
\begin{tabular}{p{4cm} p{15cm}}
Potentialstufe		& $V(x) = \begin{cases}
              		          0	& x < 0 \quad\text{Bereich A}\\
				  V	& 0 \leq x \leq D \quad\text{Bereich B}\\
				  0	& D < x \quad\text{Bereich C}
              		          \end{cases}$\\
SG			& \begin{tabular}[t]{l}
			    $\hat{H} \Psi_A(x) = -\frac{\hbar^2}{2m} \frac{d^2}{dx^2} \Psi_A(x) = E \Psi_A(x)$\\
			    $\hat{H} \Psi_B(x) = -\frac{\hbar^2}{2m} \frac{d^2}{dx^2} \Psi_B(x) + V \Psi_B(x) = E \Psi_B(x)$\\
			    $\hat{H} \Psi_C(x) = -\frac{\hbar^2}{2m} \frac{d^2}{dx^2} \Psi_C(x) = E \Psi_C(x)$
			  \end{tabular}\\
L�sungen		& \begin{tabular}[t]{ll}
			    $\Psi_A(x) = Ae^{ikx} + Be^{-ikx}$		& $k = \sqrt{\frac{2mE}{\hbar^2}}$\\
			    $\Psi_B(x) = A'e^{ik'x} + B'e^{-ik'x}$	& $k' = \sqrt{\frac{2m(E-V)}{\hbar^2}} = i\kappa$\\
			    $\Psi_C(x) = A''e^{ikx} + B''e^{-ikx}$
        		  \end{tabular}\\
Randbedingungen		& \begin{tabular}[t]{l}
			    $\Psi_A(0) = \Psi_B(0)$\\
			    $\frac{d}{dx} \Psi_A(0) = \frac{d}{dx} \Psi_B(0)$\\
			    $\Psi_B(D) = \Psi_C(D)$\\
			    $\frac{d}{dx} \Psi_B(D) = \frac{d}{dx} \Psi_C(D)$\\
			    $B'' = 0$ (keine reflektierte Komponente rechts der Barriere erlaubt)
			  \end{tabular}
\end{tabular}
Letztlich hat man nur 5 Randbedingungen f�r 6 Unbekannte. Man kann also lediglich die Tunnelwahrscheinlichkeit $P_T = \frac{|A''|^2}{|A|^2}$ angeben, indem man $A''$ durch $A$ ausdr�ckt. $P_T = \frac{4E(V-E)}{4E(V-E) + V^2\sinh^2(\kappa D)}$ Grunds�tzlich wird die Tunnelwahrscheinlichkeit klein f�r breite Barrieren ($D\to\infty$), hohe Barrieren ($V\to\infty$) und grosse Massen $m$

\subsubsection{QM harmonischer Oszillator}
Der QM harmonische Oszillator wird verwendet, um bspw. molekulare Vibrationsbewegungen zu modellieren.
\begin{tabular}{p{4cm} p{2cm}|| p{12cm}}
SG		& \multicolumn{2}{l}{$\hat{H} \Psi = \left( -\frac{\hbar^2}{2m} \frac{d^2}{dx^2} + \frac{1}{2}kx^2 \right) \Psi = E\Psi \qquad(\ast)$}\\
Transformation	& &$\lambda := \frac{2E}{\hbar\omega}\qquad x:= \sqrt{\frac{\hbar}{m\omega}}y \qquad \omega:= \sqrt{\frac{k}{m}}$\\
		& \multicolumn{2}{l}{$(\ast) \Leftrightarrow \left( \frac{d^2}{dy^2} - y^2 \right) \Psi_{\lambda} = -\lambda\Psi_{\lambda}$}\\
Absteigeoperator	& &$a := \frac{1}{\sqrt{2}} \left(y + \frac{d}{dy} \right)$\\
Aufsteigeoperator	& &$a^+ := \frac{1}{\sqrt{2}} \left(y - \frac{d}{dy} \right)$\\
			& &$2aa^+ = -\frac{d^2}{dy^2}+y^2+1$\\
			& &$2a^+a = -\frac{d^2}{dy^2}+y^2-1$\\
			& &$aa^+ = a^+a+1$\\
		& \multicolumn{2}{l}{$(\ast) = -(2aa^+-1)\Psi_{\lambda}$}\\
		& \multicolumn{2}{l}{$(\ast) \Leftrightarrow a^+a\Psi_{\lambda} = \frac{1}{2}(\lambda-1) \Psi_{\lambda}$}\\
		& \multicolumn{2}{l}{$(\ast) \Leftrightarrow aa^+\Psi_{\lambda} = \frac{1}{2}(\lambda+1) \Psi_{\lambda}$}\\
Linksmultiplikation mit $a$	& \multicolumn{2}{l}{$\underbrace{aa^+}_{=a^+a+1}a\Psi_{\lambda} = \frac{1}{2}(\lambda-1)a\Psi_{\lambda}$}\\
		& \multicolumn{2}{l}{$\Leftrightarrow a^+aa\Psi_{\lambda} = \frac{1}{2}(\lambda-3)a\Psi_{\lambda}$}\\
		& \multicolumn{2}{l}{$\Leftrightarrow a^+a \Psi_{\lambda-2} = \frac{1}{2}(\lambda-3) \Psi_{\lambda-2}$}\\
Konsequenz	& \multicolumn{2}{l}{$a\Psi_{\lambda} ~\propto ~\Psi_{\lambda-2}$ und $a^+\Psi_{\lambda} ~\propto ~\Psi_{\lambda+2}$}\\
$\lambda_{min}$	& \multicolumn{2}{l}{$0 = a^+a\Psi_{\lambda_{min}} = \frac{1}{2}(\lambda_{min}-1) \Psi_{\lambda_{min}}$}\\
		& \multicolumn{2}{l}{$\Rightarrow \lambda_{min} = 1$ Mit dem Aufsteigeoperator erh�lt man weitere $\lambda = 1,3,5,7,...$}\\
Eigenwerte	& \multicolumn{2}{l}{$\boldsymbol{E \mu = (\mu + \frac{1}{2})\hbar\omega}, \quad \mu = 0,1,2,...\quad \text{und }\lambda = 2\mu + 1\quad \omega = \sqrt{\frac{k}{m}}$}\\
Eigenfunktionen	& \multicolumn{2}{l}{$\boldsymbol{\Psi_{\mu} = N_{\mu}H_{\mu}(\sqrt{\alpha}x) e^{-\alpha x^2/2 }}\quad \alpha = \frac{\sqrt{mk}}{\hbar}$}
\end{tabular}
\subsubsection{3D QM harm. Osz.}
\begin{tabular}{p{4cm} p{15cm}}
Eigenwerte	& $E_n = (n_x + n_y + n_z + \frac{3}{2})\hbar\omega$\\
Eigenfunktionen	& $\Psi_n = \Psi_{n_x} \cdot \Psi_{n_y} \cdot \Psi_{n_z}$
\end{tabular}
\subsection{Wellengruppen}
\begin{tabular}{p{4cm} p{15cm}}
L�sungen der zeitabh. SG	& $\Psi_n(q_i,t) = \varphi_n(q_i)e^{-iE_nt/\hbar}$\\
WSKDichte			& Die WSKdichte $|\Psi_n(q_i,t)|^2$ ist zeitunabh�ngig, da\\
				& $|\Psi_n(q_i,t)|^2 = \varphi_n^{\ast}(q_i)e^{iE_nt/\hbar} \cdot \varphi_n(q_i)e^{-iE_nt/\hbar} = |\varphi_n(q_i)|^2$\\
allg. L�sung der SG		& $\Psi(q_i,t) = \sum_n c_n \varphi_n(q_i)e^{-iE_nt/\hbar}$
\end{tabular}