\section{QM Drehimpulse}
\begin{tabular}{ll}
Symbol		& Drehimpuls\\\hline
$\vec{l}$	& Bahndrehimpuls eines Elektrons\\
$\vec{s}$	& Spin eines Elektrons\\
$\vec{j}$	& Gesamtdrehimpuls eines Elektrons\\
$\vec{L}$	& Gesamtbahndrehimpuls eines Atoms oder Molek�ls\\
$\vec{S}$	& Gesamtelektronenspin eines Atoms oder Molek�ls\\
$\vec{J}$	& Gesamtdrehimpuls ohne Kernspin\\
$\vec{I_i}$	& Kernspin des i-ten Kerns eines Molek�ls\\
$\vec{F}$	& Gesamtdrehimpuls
\end{tabular}
\subsection{Quantenzahlen}
\begin{tabular}[t]{p{3cm}lll|lll}
			& \multicolumn{3}{l}{klassisch, via Korrespondenzprinzip}	& \multicolumn{3}{l}{QM, allgemein via Kommutatoren}\\
Hauptquantenzahl	& n & 1,2,3,... & unlimitiert\\
			&   & K,L,M,... &\\
Drehimpuls- quantenzahl	& l & 0,1,2,... n-1 & n m�gliche Werte			& J & 0, $\tfrac{1}{2}$,1, $\tfrac{3}{2}$,2,...\\
			&   & s,p,d,f,..    &\\
magnetische Quantenzahl	& $m_l$ & -l, l+1,..., l-1, l& 2l+1 m�gliche Werte	& M & -J,-J+1,...,J-1,J	& 2J+1 m�gliche Werte\\
\end{tabular}
\subsection{Bahndrehimpuls}
\begin{tabular}{p{4cm} p{15cm}}
Klassische Definition		& $\vec{l} = \vec{r} \times \vec{p}$\\
				& $|\vec{l}| = |\vec{r}||\vec{p}| \sin\alpha$\\
QM Definition			& $\hat{\vec{l}} = -i\hbar
					\begin{pmatrix}
					y\frac{\partial}{\partial z} - z\frac{\partial}{\partial y}\\ 
					z\frac{\partial}{\partial x} - x\frac{\partial}{\partial z}\\ 
					x\frac{\partial}{\partial y} - y\frac{\partial}{\partial x}
					\end{pmatrix}$\\
				& $|\vec{l}|^2 = \hat{l}^2 = \hat{l}_x^2 + \hat{l}_y^2 + \hat{l}_z^2$\\
Vertauschungsrelationen		& $[\hat{l}_x, \hat{l}_y] = i\hbar \hat{l}_z, \quad [\hat{l}_x, \hat{l}_z] = i\hbar \hat{l}_y, \quad[\hat{l}_y, \hat{l}_z] = i\hbar \hat{l}_x$\\
				& $[\hat{l}^2, \hat{l}_x] = 0, \quad [\hat{l}^2, \hat{l}_y] = 0, \quad [\hat{l}^2, \hat{l}_z] = 0$\\
				& Gem�ss Thm. 4 haben also $\hat{l}^2$ und $\hat{l}_z$ eine gemeinsame Basis von Eigenfunktionen:\\
Gel�ste Eigenwertgleichungen	& \begin{tabular}[t]{|lll|}
						  \hline
						  $\hat{l}^2 Y(\theta,\phi) = \hbar^2 l(l+1)  Y(\theta,\phi)$	& $l = 0,1,2,...$		& $l$: Bahndrehimpulsquantenzahl\\
                                             	  $\hat{l}_z Y(\theta,\phi) = \hbar m Y(\theta,\phi)$		& $m = -l,-l+1,...,l-1,l$	& $m$: magnetische Quantenzahl\\\hline
                                             	  \end{tabular}\\
Bra-Ket Notation		& $Y_{l,m} = |l,m \rangle$\\
Unbestimmtheitsrelation		& $\Delta l_x \Delta l_y \geq \frac{1}{2} |\langle [\hat{l}_x, \hat{l}_y] \rangle | = \frac{1}{2} \hbar | \langle \hat{l}_z \rangle |$\\
Spherical Harmonics		& $Y_{l,m} (\theta,\phi) = N_{l,m} P_l^{|m|} (\cos \theta) e^{im\phi}, \text{wobei } N_{l,m} = \sqrt{\frac{(2l+1)(l-|m|)!}{4\pi(l+|m|)!}}$\\
\textbf{Parit�t von $Y_{l,m}$}	& $\boxed{\text{Parit�t}(Y_{l,m}) = (-1)^l}$\\
Legendre-Polynome		& \begin{tabular}[t]{l}
				    $P_n^m(z) \equiv (1-z^2)^{m/2} \frac{d^m}{dz^m}P_n(z)$\\
				    $P_n(z) = \frac{ 1}{2^n n!}\frac{d^n}{dz^n}(z^2-1)^n$\\
				    $P_0^0 = 1, ~~ P_1^0 = z, ~~ P_1^1 = (1-z^2)^{1/2}$\\
				    $P_2^0 = \frac{1}{2} (3z^2-1), ~~ P_2^1 = 3(1-z^2)^{1/2}\cdot z, ~~ P_2^2 = 3(1-z^2)$
				  \end{tabular}\\
\end{tabular}
\subsection{Allgemeine Drehimpulse}
Allg. Drehimpulse $J$ werden rein quantenmechanisch mit Vertauschungsrelationen definiert, es existiert f�r sie kein klassisches Analogon. Erkl�rt werden kann damit der Spin.\\
\begin{tabular}{p{4cm} p{15cm}}
Vertauschungsrelationen		& $[\hat{J}_x, \hat{J}_y] = i\hbar \hat{J}_z, \quad [\hat{J}_x, \hat{J}_z] = i\hbar \hat{J}_y, \quad[\hat{J}_y, \hat{J}_z] = i\hbar \hat{J}_x$\\
				& $[\hat{J}^2, \hat{J}_x] = 0, \quad [\hat{J}^2, \hat{J}_y] = 0, \quad [\hat{J}^2, \hat{J}_z] = 0$\\
Gel�ste Eigenwertgleichungen	& \begin{tabular}[t]{|lll|}
				    \hline
				    $\hat{J}^2 \Psi = \hbar^2 J(J+1)  \Psi$	& $J = 0,\frac{1}{2},1,\frac{3}{2},...$	& $J$: Drehimpulsquantenzahl\\
                                    $\hat{J}_z \Psi = \hbar M \Psi$		& $M = -J, -J+1,...,J-1,J$		& $M$: magnetische Quantenzahl\\\hline
                                  \end{tabular}\\
Bra-Ket Notation		& $\Psi_{J,M} = |J,M \rangle$\\
Orientierung von Drehimpulsvektoren& \begin{tabular}[t]{lll}
				    $|\vec{J}| = \hbar \sqrt{J(J+1)}$	& L�nge		& Entartung: $2J+1$\\
				    $J_z = \hbar M$			& z-Komponente	& Entartung: 1
				  \end{tabular}
\end{tabular}
\subsection{Matrixdarstellung von Drehimpulsoperatoren}
\begin{tabular}{p{4cm} p{15cm}}
Matrixdarstellung eines Operators	& $A_{nm} = \int \Psi_n^{\ast} \hat{A} \Psi_m = \langle n | \hat{A} | m \rangle$ Achtung! Nicht mit Erwartungswert verwechseln!\\
Aufsteigeoperator			& $\hat{J}_+ = \hat{J}_x + i\hat{J}_y$\\
Absteigeoperator			& $\hat{J}_- = \hat{J}_x - i\hat{J}_y$\\
					& $\hat{J}_x = \frac{\hat{J}_+ + \hat{J}_-}{2}$\\
					& $\hat{J}_y = \frac{\hat{J}_+ - \hat{J}_-}{2i}$\\
$\hat{J}_z$				& $\langle J',M' | \hat{J}_z | J,M \rangle = \hbar M \langle J',M' | J,M \rangle = \hbar M \delta_{J',J} \delta_{M',M}$\\
$\hat{J}^2$				& $\langle J',M' | \hat{J}^2 | J,M \rangle = \hbar^2 J(J+1) \langle J',M' | J,M \rangle = \hbar M \delta_{J',J} \delta_{M',M}$\\
$\hat{J}_+$				& $\langle J',M' | \hat{J}_+ | J,M \rangle = \hbar \sqrt{J(J+1)-M(M+1)} \delta_{J',J} \delta_{M',M+1}$\\
$\hat{J}_-$				& $\langle J',M' | \hat{J}_- | J,M \rangle = \hbar \sqrt{J(J+1)-M(M-1)} \delta_{J',J} \delta_{M',M-1}$\\
\end{tabular}
\subsubsection{Pauli-Matrizen}
Beispiel f�r eine Matrixdarstellung f�r $J = \frac{1}{2} \Rightarrow M = \pm \frac{1}{2}$. (Spin 1/2 Teilchen)\\
Es gibt also nur zwei Eigenfunktionen: $|\tfrac{1}{2}, \tfrac{1}{2} \rangle = : |\alpha \rangle$ und $|\tfrac{1}{2}, -\tfrac{1}{2} \rangle =: |\beta\rangle$\\
\begin{tabular}{p{4cm} p{15cm}}
$\hat{J}_z$	& $\hat{J}_z = \begin{bmatrix} \langle \alpha | \hat{J}_z | \alpha \rangle 	& \langle \alpha | \hat{J}_z | \beta \rangle \\
				\langle \beta | \hat{J}_z | \alpha \rangle 			& \langle \beta | \hat{J}_z | \beta \rangle \end{bmatrix}
			      = \frac{\hbar}{2} \begin{bmatrix}1 & 0\\0 & -1\end{bmatrix} = \frac{\hbar}{2} \sigma_z$\\
$\hat{J}^2$	& $\hat{J}^2 = \hbar^2 \begin{bmatrix} \tfrac{3}{4} & 0 \\ 0 & \tfrac{3}{4} \end{bmatrix} = \frac{3\hbar^2}{4} \mathbb{I}_2$\\
$\hat{J}_x$	& $\hat{J}_x = \frac{\hat{J}_+ + \hat{J}_-}{2} = \frac{\hbar}{2} \begin{bmatrix}0 & 1 \\ 1 & 0\end{bmatrix} = \frac{\hbar}{2} \sigma_x$\\
$\hat{J}_y$	& $\hat{J}_y = \frac{\hat{J}_+ + \hat{J}_-}{2i} = \frac{\hbar}{2} \begin{bmatrix}0 & -i \\ i & 0\end{bmatrix} = \frac{\hbar}{2} \sigma_y$\\
\end{tabular}




