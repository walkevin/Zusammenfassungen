\section{Born-Oppenheimer-N�herung f�r Molek�le}
Problem: F�r ein Molek�l mit $K$ Kernen und $N$ Elektronen ist das zu l�sende Problem (3N + 3K)-dimensional. Im Folgenden bezeichnen $Q$, resp. $q$ die Gesamtheit aller Kerne bzw. Elektronen.\\
\begin{tabular}{p{4cm} p{15cm}}
Ansatz		& $\Psi(\vec{q},\vec{Q}) = \underbrace{\Phi(\vec{q},\vec{Q})}_{\text{Elektronen}} \underbrace{\varphi(\vec{Q})}_{\text{Kerne}}$\\
Elektronenbewegung	& SG: $\hat{H}_e \Phi_n(q,Q) = U_n(Q)\Phi_n(q,Q)$\\
Born-Oppenheimer Hyperfl�che	& $U_n(Q)$\quad (Energiefunktion)\\
Dimension der Born-Oppenheimer Hyperfl�che	& \begin{tabular}[t]{ll}
						    lineare Molek�le		& $3K-5$\\
						    nichtlineare Molek�le	& $3K-6$
						  \end{tabular}\\
						& $K:$ Anzahl Kerne. $U_n$ ist unabh�ngig von Position des Molek�ls (3 Dim.) und von dessen Orientierung im Raum (2 Dim. falls linear, 3 Dim. falls nichtlinear)\\
Bermerkungen					& \begin{itemize}
            					  	\item $U_n$ h�ngt nicht von den Kernmassen, sondern nur von deren Ladungen ab.
							\item Stabile Kernkonfigurationen entsprechen lokalen Minima auf der Born-Oppenheimer-Hyperfl�che.
            					  \end{itemize}
\end{tabular}

