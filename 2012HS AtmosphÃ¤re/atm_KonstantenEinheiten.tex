\section{Anhang}
\subsection{N�tzliche Konstanten}
\begin{tabular}{p{7cm} p{3cm} p{8cm}}
Konstante			& Symbol	& Wert\\\midrule
Avogadro-Konstante		& $N_A$		& $6.0221 \cdot 10^{23} $mol$^{-1}$\\
Boltzmann-Konstante		& $k_B$		& $1.3807 \cdot 10^{-23} $J K$^{-1}$\\
Molvolumen f�r 1 mol Gas unter Normbedingungen	& $V$	& $22.41$ l\\
Molmasse f�r trockene Luft	& $M_d$		& $28.96 $g mol$^{-1}$ (Standardatmosph�re)\\
universelle Gaskonstante	& $R$		& $8.314$ J mol$^{-1}$ K$^{-1}$\\
Gaskonstante f�r trockene Luft	& $R_d = R/M_d$	& $287$ J kg$^{-1}$ K$^{-1}$\\
Dichte f�r 1 mol trockene Luft unter Normbedingungen	& $\varrho_d = nM_d / V$	& $1292.12$ g m$^{-3}$\\
spezifische W�rmekapazit�t von Luft\\
- bei konstantem Volumen	& $c_V$		& $1005$ J kg$^{-1}$ K$^{-1}$\\
- bei konstantem Druck		& $c_p$		& $717$ J kg$^{-1}$ K$^{-1}$\\
Schwerebeschleunigung		& $g$		& \begin{tabular}[t]{lll}
						    auf Meeresniveau	& am �quator:	& $9.780$ m s$^{-2}$\\
									& am Pol:	& $9.832$ m s$^{-2}$\\
						    in 30 km H�he	& am �quator:	& $9.689$ m s$^{-2}$\\
									& am Pol:	& $9.740$ m s$^{-2}$
						  \end{tabular}\\
Winkelgeschw. der Erdrotation	& $\Omega$	& $7.292 \cdot 10^{-5}$ s$^{-1}$
\end{tabular}
\subsection{Normbedingungen}
\begin{tabular}{p{5cm}p{14cm}}
Temperatur	& $T$ = 273.15 K\\
Druck		& $p$ = 1013.25 hPa
\end{tabular}

\subsection{N�tzliche Fakten}
\begin{tabular}{p{5cm}p{14cm}}
Temperatur			& K = 273.15 + $^{\circ}$C\\
Bar - Pascal			& 1 bar = $10^5$ Pa\\
Sonnentemperatur		& T$_{Sonne} = 5780$ K\\
Erdtemperatur			& T$_{Erde} = 290$ K
\end{tabular}
\subsection{Einheiten und Zeichen}
\begin{tabular}{p{8cm} p{2cm} p{4cm} p{4cm}}
Gr�sse			& Symbol	& Bezeichnung der Einheit	& SI-Gr�sse\\\midrule
L�nge			& $l$		& Meter				& m\\
Masse			& $m$		& Kilogramm			& kg\\
Zeit			& $t$		& Sekunde			& s\\
Geschwindigkeit		& $v$		&				& ms$^{-1}$\\
Beschleunigung		& $a$		&				& ms$^{-2}$\\
Impuls			& $p$		&				& m kg s$^{-1}$\\
Kraft			& $F$		& Newton (N)			& m kg s$^{-2}$\\
Druck			& $p$		& Pascal (Pa)			& m$^{-1}$ kg s$^{-2}$\\
Arbeit			& $W$		& Joule (J)			& m$^2$ kg s$^{-2}$\\
Leistung		& $P$		& Watt (W)			& m$^2$ kg s$^{-3}$\\
Energie			& $U,E$		& Joule (J)			& m$^2$ kg s$^{-2}$\\
Temperatur		& $T$		& Kelvin (K)			& m$^2$ kg s$^{-2}$ / Teilchen\\
\end{tabular}