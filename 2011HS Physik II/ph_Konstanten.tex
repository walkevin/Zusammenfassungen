\section{Anhang}
\subsection{Grundkonstanten}
\begin{tabular}{p{8cm} p{2cm} p{8cm}}
Konstante			& Symbol	& Wert\\\midrule
Lichtgeschwindigkeit im Vakuum	& $c$		& $2.9979 \cdot 10^8 $m s$^{-1}$\\
Elementarladung			& $e$		& $1.6021 \cdot 10^{-19} $C\\
Ruhemasse des Elektrons		& $m_e$		& $9.1094 \cdot 10^{-31}$kg\\
Ruhemasse des Protons		& $m_p$		& $1.6726 \cdot 10^{-27}$kg\\
Ruhemasse des Neutrons		& $m_n$		& $1.6749 \cdot 10^{-27}$kg\\
Planck'sche Konstante		& $h$		& $6.621 \cdot 10^{-34} $J s\\
				&		& $= 4.135 \cdot 10^{-15} $eV s\\
				& $\hbar = \frac{h}{2\pi}$	& $1.0546 \cdot 10^{-34}$ J s\\
Bohr'scher Radius		& $a_0$		& $5.2918 \cdot 10^{-11}$m\\
Comptonwellenl�nge		&\\
$\quad$ d. Elektrons		& $\lambda_{c,e}$	& $2.4263 \cdot 10^{-12}$m\\
$\quad$ d. Protons		& $\lambda_{c,p}$	& $1.3214 \cdot 10^{-15}$m\\
Rydberg-Konstante		& $R$		& $1.0974 \cdot 10^7 $m$^{-1}$\\
Bohr'sches Magneton		& $\mu_B$	& $9.2740 \cdot 10^{-24} $J T$^{-1}$\\
Avogadro-Konstante		& $N_A$		& $6.0221 \cdot 10^{23} $mol$^{-1}$\\
Boltzmann-Konstante		& $k_B$		& $1.3807 \cdot 10^{-23} $J K$^{-1}$\\
				&		& $= 8.6173 \cdot 10^{-5} $eV K$^{-1}$\\
universelle Gaskonstante	& $R$		& $8.314$ J mol$^{-1}$ K$^{-1}$\\
Stefan-Boltzmann-Konstante	& $\sigma$	& $5.67\cdot 10^{-8} $W m$^{-2}$K$^{-4}$\\
Schwerebeschleunigung in Meeresh�he am �quator	& $g$	& $9.7805 $m s$^{-2}$\\
Atommasse			& $u$		& $1.660 \cdot 10^{-27}$ kg
\end{tabular}
\subsection{Einheiten und Zeichen}
\begin{tabular}{p{8cm} p{2cm} p{4cm} p{4cm}}
Gr�sse			& Symbol	& Bezeichnung der Einheit	& SI-Gr�sse\\\midrule
L�nge			& $l$		& Meter				& m\\
Masse			& $m$		& Kilogramm			& kg\\
Zeit			& $t$		& Sekunde			& s\\
Geschwindigkeit		& $v$		&				& ms$^{-1}$\\
Beschleunigung		& $a$		&				& ms$^{-2}$\\
Winkelgeschwindigkeit	& $\omega$	&				& s$^{-1}$\\
Kreisfrequenz		& $\omega$	&				& s$^{-1}$\\
Frequenz		& $\nu$		& Hertz (Hz)			& s$^{-1}$\\
Impuls			& $p$		&				& m kg s$^{-1}$\\
Kraft			& $F$		& Newton (N)			& m kg s$^{-2}$\\
Druck			& $p$		& Pascal (Pa)			& m$^{-1}$ kg s$^{-2}$\\
Drehimpuls		& $L$		&				& m$^2$ kg s$^{-1}$\\
Drehmoment		& $\tau$	&				& m$^2$ kg s$^{-2}$\\
Arbeit			& $W$		& Joule (J)			& m$^2$ kg s$^{-2}$\\
Leistung		& $P$		& Watt (W)			& m$^2$ kg s$^{-3}$\\
Energie			& $U,E$		& Joule (J)			& m$^2$ kg s$^{-2}$\\
Temperatur		& $T$		& Kelvin (K)			& m$^2$ kg s$^{-2}$ / Teilchen\\
Diffusionskoeffizient	& $D$		&				& m s$^{-2}$\\
W�rmeleitf�higkeit	& $K$		&				& m kg s$^{-3}$ K$^{-1}$\\
Elektrischer Strom	& $I$		& Amp�re (A)			& A\\
Ladung			& $q,Q$		&				& A s\\
Elektrisches Feld	& $\epsilon$	& 				& m kg s$^{-3}$ A$^{-1}$\\
Elektrisches Potential	& $V$		& Volt (V)			& m$^2$ kg s$^{-3}$ A$^{-1}$\\
Stromdichte		& $j$		&				& m$^{-2}$ A\\
Magnetfeld		& $B$		& Tesla (T)			& kg s$^{-2}$ A$^{-2}$\\
\end{tabular}
\subsection{N�tzliche Beziehungen}
\begin{tabular}{p{5cm}p{14cm}}
Elektronenvolt			& $1 $eV$ = 1.602 \cdot 10^{-19}$ J\\
				& $1 $J$ = 6.242 \cdot 10^{18}$ eV\\
Planck'sches Wirkungsquantum	& $hc = 1240 $eV$\cdot nm$\\
Boltzmann-Konstante		& $k_B T = 25.8 $ meV $ ($at $T=300 K) = 4.133 \cdot 10^{-21}J$\\
Temperatur			& K = 273.15 + $^{\circ}$C\\
Bar - Pascal			& 1 bar = $10^5$ Pa\\
Coulomb - Ampere		& C = A s
\end{tabular}\\
\begin{tabular}{lcl|p{15cm}}
Energie $E = \hbar\omega$	& $\Leftrightarrow$	& Wellenl�nge $\lambda$	& $E [eV] = \frac{1.24}{\lambda[\mu m]}$\\
Frequenz $\nu$			& $\Leftrightarrow$	& reduzierte Wellenzahl $\frac{1}{\lambda}$	& $\nu[GHz] = \frac{30}{\lambda}\left[cm^{-1}\right]$\\
Kreisfrequenz $\omega$		& $\Leftrightarrow$	& Wellenzahl $k$	& $\omega[GHz] = 30 \cdot k \left[cm^{-1}\right]$
\end{tabular}
\subsection{Koordinatensysteme}
\subsubsection{Polarkoordinaten}
\begin{tabular}{p{4cm} p{15cm}}
Definition	& $\begin{array}[t]{l}
                	  x = r \cdot \sin \varphi\\
			  y = r \cdot \cos \varphi\\
                	 \end{array}$\\
Gradient	& $\nabla f= \frac{\partial f}{\partial r } \vec{e}_r +  \frac{1}{r} \frac{\partial f}{\partial \varphi} \vec{e}_\varphi$\\
Laplace		& $\Delta f(r, \varphi ) =
\frac{\partial^2 f}{\partial r^2} +
\frac{1}{r}\frac{\partial f}{\partial r} +
\frac{1}{r^2}\frac{\partial^2 f}{\partial \varphi^2}$\\
oder		& $\Delta f(r, \varphi ) =
\frac{1}{r}\frac{\partial}{\partial r}
\left( r\,\frac{\partial f}{\partial r} \right) +
\frac{1}{r^2}\frac{\partial^2 f}{\partial \varphi^2}$\\
Volumenelement	& d$x$ d$y$ = $r$ d$\varphi$ d$r$
\end{tabular}
\subsubsection{Zylinderkoordinaten}
\begin{tabular}{p{4cm} p{15cm}}
Definition	& $\begin{array}[t]{l}
                	  x = r \cdot \sin \varphi\\
			  y = r \cdot \cos \varphi\\
			  z = z
                	 \end{array}$\\
Gradient	& $\nabla f= \frac{\partial f}{\partial r } \vec{e}_r +  \frac{1}{r} \frac{\partial f}{\partial \varphi} \vec{e}_\varphi+ \frac{\partial f}{\partial z} \vec{e}_z $\\
Laplace		& $\Delta f ( r , \varphi , z ) = \frac{1}{r} \frac{\partial}{\partial r}
\left( r\,\frac{\partial f}{\partial r} \right) +
\frac{1}{r^2}\frac{\partial^2 f}{\partial \varphi^2} +
\frac{\partial^2 f}{\partial z^2}$\\
Volumenelement	& d$x$ d$y$ d$z$ = $r$ d$z$ d$\varphi$ d$r$
\end{tabular}
\subsubsection{Kugelkoordinaten}
\begin{tabular}{p{4cm} p{15cm}}
Definition		&$\begin{array}[t]{l}
                	  x = r \cdot \sin \theta \cdot \cos \varphi\\
			  y = r \cdot \sin \theta \cdot \sin \varphi\\
			  z = r \cdot \cos \theta
                	 \end{array}$\\
Gradient		& $\mathbf{\nabla}
  =\mathbf{e}_r\frac{\partial}{\partial r} 
   + \mathbf{e}_\theta\frac{1}{r}\frac{\partial}{\partial\theta}
   + \mathbf{e}_\varphi\frac{1}{r\sin\theta}\frac{\partial}{\partial\varphi}$\\
Laplace			& $\Delta f ( r , \theta , \varphi ) = \frac{1}{r^2} 
\frac{\partial}{\partial r} \left( r^2  \,\frac{\partial f}{\partial r} \right) +
\frac{1}{r^2 \sin \theta}  \frac{\partial}{\partial \theta} \left(\sin\theta \, \frac{\partial f}{\partial \theta} \right) +
\frac{1}{r^2 \sin^2\theta}  \frac{\partial^2 f}{\partial \varphi^2}$\\
Volumenelement		& d$x$ d$y$ d$z$ = $r^2 \sin \theta $ d$\varphi$ d$\theta$d$r$
\end{tabular}

\subsection{Trigonometrische Identit�ten und Integrale}
\subsubsection{Identit�ten}
\begin{tabular}{ll}
$\sin(a \pm b) = \sin a \cos b \pm \cos a \sin b$	& $\sin a \sin b = \frac{1}{2} \left( \cos(a-b) - \cos (a + b) \right)$\\
$\cos(a \pm b) = \cos a \cos b \mp \sin a \sin b$	& $\cos a \cos b = \frac{1}{2} \left( \cos(a-b) + \cos (a + b) \right)$\\
$\sin^2(x) = -\frac{\cos(2x)}{2} + \frac{1}{2}$		& $\sin a \cos b = \frac{1}{2} \left( \sin(a-b) + \sin (a + b) \right)$\\
$\cos^2(x) = \frac{\cos(2x)}{2} + \frac{1}{2}$		& $\sin(2x) = 2\sin(x) \cos(x)$\\
							& $\cos(2x) = \cos^2(x)-\sin^2(x)$
\end{tabular}

\subsubsection{Integrale}
$
\begin{array}{ll}
\int \sin^2(ax) dx = \frac{x}{2}  - \frac{1}{4a} \sin(2ax)			& \int \sin^3(ax) dx = -\frac{1}{a} \cos(ax) + \frac{1}{3a} \cos^3(ax)\\
\int x \cdot \sin(ax) dx = \frac{1}{a^2} \sin(ax) - \frac{x}{a} \cos(ax)	& \int x^2 \sin(ax) dx = \frac{2x}{a^2} \sin(ax) - \frac{x^2}{a} \cos(ax) + \frac{2}{a^3}\cos(ax)\\
\int \cos^2(ax) dx = \frac{x}{2}  + \frac{1}{4a} \sin(2ax)			& \int \cos^3(ax) dx = -\frac{1}{a} \cos(ax) - \frac{1}{3a} \sin^3(ax)\\
\int x \cdot \cos(ax) dx = \frac{1}{a^2} \cos(ax) + \frac{x}{a} \sin(ax)	& \int x^2 \cos(ax) dx = \frac{2x}{a^2} \cos(ax) + \frac{x^2}{a} \sin(ax) - \frac{2}{a^3}\sin(ax)\\
\int x \cdot \sin^2(ax) dx = -\frac{2ax (\sin(2ax)-ax) + \cos(2ax)}{8a^2}

\end{array}
$
\subsection{Eigenschaften gerader und ungerader Funktionen}
\begin{itemize}
\item $f(g(x))$ ist gerade, falls $g(x)$ gerade ist. $f(x)$ beliebig. Z.B. $\sin(x^2)$
\item $f(g(x))$ ist gerade, falls $g(x)$ ungerade und $f(x)$ gerade. Z.B. $\sin^2(x)$
\item $f(g(x))$ ist ungerade, falls $g(x)$ ungerade und $f(x)$ ungerade. Z.B. $\sin^3(x)$
\item Cosinus ist gerade, Sinus ist ungerade.
\item Die Ableitung einer geraden Funktion ist ungerade, die Ableitung einer ungeraden Funktion ist gerade.
\item Die Fourierreihe einer geraden (ungeraden) Funktion enth�lt nur Cosinus - (Sinus-) Terme.
\end{itemize}

\subsection{Genaue Funktionswerte}
$
\begin{array}{l|ccccccccc}
 \alpha		& 0	& \frac{\pi}{6}		& \frac{\pi}{4}		& \frac{\pi}{3}		& \frac{\pi}{2}	& \frac{2\pi}{3}	& \frac{3\pi}{4}	& \frac{5\pi}{6}	& \pi\\\hline
 \sin\alpha	& 0	& \frac{1}{2}		& \frac{\sqrt{2}}{2}	& \frac{\sqrt{3}}{2}	& 1		& \frac{\sqrt{3}}{2}	& \frac{\sqrt{2}}{2}	& \frac{1}{2}		& 0\\
 \cos\alpha	& 1	& \frac{\sqrt{3}}{2}	& \frac{\sqrt{2}}{2}	& \frac{1}{2}		& 0		& -\frac{1}{2}		& -\frac{\sqrt{2}}{2}	& -\frac{\sqrt{3}}{2}	& -1\\
 \tan\alpha	& 0	& \frac{\sqrt{3}}{3}	& 1			& \sqrt{3}		& -		& -\sqrt{3}		& -1			& -\frac{\sqrt{3}}{3}	& 0\\
\end{array}\\
$

\subsection{Fourierreihen}
\subsubsection{Definition}
 Lipstetige T-periodische Funktionen $( f(t) = f(t+T) )$ lassen sich als trigonometrische Reihe darstellen
\begin{equation*}
 f(t) = \frac{a_0}{2} + \sum_{n=1}^{\infty} \left(a_n \cos\left(\frac{2\pi}{T} nt \right) + b_n \sin\left(\frac{2\pi}{T} nt \right) \right)
\end{equation*}
mit reellen Koeffizienten $a_n, b_n$ oder wie folgt:
\begin{equation*}
 f(t) = \sum_{n=-\infty}^{\infty} c_n e^{\frac{2\pi i}{T} nt},\quad c_n\in\mathbb{C}
\end{equation*}
mit komplexen Koeffizienten $c_n$\\
\subsubsection{Bestimmung der Koeffizienten}
$c_n = \frac{1}{2} (a_n -ib_n)\\
c_{-n} = \frac{1}{2} (a_n +ib_n) = \overline{c_n}\\
c_0 = \frac{1}{2} a_0\\\\
c_n = \frac{1}{T} \int_{-T/2}^{T/2} f(t)\cdot e^{\frac{-2\pi i}{T}nt} dt\\
a_n = \left(c_n + c_{-n}\right) = \frac{2}{T} \int_{-T/2}^{T/2} f(t) \cos\left(\frac{2\pi}{T}nt \right) dt\\
b_n = \left(\frac{c_n - c_{-n}}{i}\right) =  \frac{2}{T} \int_{-T/2}^{T/2} f(t) \sin\left(\frac{2\pi}{T}nt \right) dt$\\\\
Falls $f$ gerade $\Leftrightarrow f(t) = f(-t)$, dann $b_n = 0 \forall n$\\
Falls $f$ ungerade $\Leftrightarrow f(-t) = -f(t)$, dann $a_n = 0 \forall n$\\

\subsection{Fouriertransformation (FT)}
 \begin{align*}
 \text{FT} := &\hat{f}(\omega) = \int_{-\infty}^{\infty} f(t)e^{-i\omega t} dt\quad \omega\in\mathbb{R}\\
	  &f(t) = \frac{1}{2\pi} \int_{-\infty}^{\infty} \hat{f}(\omega) e^{i\omega t} d\omega
\end{align*}
$
\begin{array}{lll}
 \text{Linearit�t:} 	&h(t) = af(t) + bg(t)			&\Rightarrow \hat{h}(t) = a\hat{f}(t) + b\hat{g}(t)\\
 \text{Verschiebung:}	&g(t) = f(t-a)				&\Rightarrow \hat{g}(\omega) = e^{-ia\omega} \hat{f}(\omega)\\
 			&g(t) = e^{iat}f(t)			&\Rightarrow \hat{g}(\omega) = \hat{f}(\omega-a)\\
 \text{Skalierung:}	&g(t) = f\left(\frac{t}{a}\right) (a>0)	&\Rightarrow \hat{g}(\omega) = a\hat{f}(a\omega)\\
 \text{Ableitung:} 	&g(t) = f'(t)				&\Rightarrow \hat{g}(\omega) = i\omega\hat{f}(\omega)\\
 \text{Faltungsprodukt:}& \multicolumn{2}{l}{\text{Definition im ``Fourierraum'':} \Bigl( f\ast g\Bigr)(t) := \int_{-\infty}^{\infty} f(t-s)g(s) ds}\\
			&\Bigl(f\ast g\Bigr)(t)			&\Rightarrow \widehat{f\ast g} (\omega) = \hat{f}(\omega)\cdot \hat{g}(\omega)\\
 \text{Standard-FT:}	&f(t) = e^{-\frac{at^2}{2}}\quad a>0	&\Rightarrow \hat{f}(\omega) = \sqrt{\frac{2\pi}{a}} e^{\frac{-\omega^2}{2a}}
\end{array}
$