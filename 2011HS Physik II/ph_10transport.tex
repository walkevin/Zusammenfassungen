\section{Transportph�nomene}
\begin{tabular}[t]{p{4cm}p{15cm}}
mikroskopischer Stossquerschnitt	& $\sigma = 4\pi r^2$\\
mittlere freie Wegl�nge			& $\Lambda = \frac{1}{n\sigma}\quad n:$ Teilchendichte\\
``Teilchenintensit�t''			& $W(L) = ae^{-L/\Lambda}$\\
Absorptionsl�nge			& $l_a \equiv \Lambda_a = \frac{1}{n\sigma_a}$\\
Absorptionskoeffizient			& $\alpha = n\sigma_a\quad$ Anzahl absorbierte Photonen pro L�ngeneinheit\\
Intensit�t des Lichts			& $I(z) = I_0e^{-\alpha z}$
\end{tabular}
\subsection{Diffusion und W�rmeleitung}
\begin{tabular}{p{9.5cm}p{9.5cm}}
\textbf{Diffusion}						& \textbf{W�rmeleitung}\\\midrule
Fick'sches Gesetz (Diffusionsstromdichte)			& Fourier'sches Gesetz (W�rmestromdichte)\\
$\mathbf{j(r)} = -D \textrm{~grad~} n(\boldsymbol{r})$		& $\boldsymbol{j_Q(r)} = -K \textrm{~grad~} T(\boldsymbol{r})$\\\midrule
Diffusionskonstante $[D] = \frac{m^2}{s}$			& W�rmeleitf�higkeit $[K] = \frac{J}{m\cdot s \cdot K} = \frac{W}{m\cdot K}$\\\midrule
f�r Gase und Fl�ssigkeiten					& f�r Gase und Fl�ssigkeiten\\
$D = \frac{1}{3}\langle v \rangle \Lambda$			& $K = D\left( \frac{3}{2}nk_B \right) = \frac{1}{2}nk_B \langle v \rangle \Lambda$\\
$\langle v \rangle = \sqrt{\frac{8 k_BT}{\pi m}}$		& \\\midrule
station�rer Zustand: $j(x) =$ konst.				& station�rer Zustand: $j_Q(x) =$ konst.\\
$n(x) = -\frac{j}{D}x + n_0$					& $T(x) = -\frac{j_Q}{K}x + T_0$\\\midrule
Kontinuit�tsgleichung: Teilchenerhaltung			& Kontinuit�tsgleichung: Energieerhaltung\\
$\frac{\partial n}{\partial t} + \textrm{~div~} \boldsymbol{j} = 0$	& $\frac{\partial \rho_Q}{\partial t} + \textrm{~div~} \boldsymbol{j_Q} = 0$\\\midrule
\textbf{Diffusionsgleichung}					& \textbf{W�rmeleitungsgleichung}\\
$\frac{\partial n(x,t)}{\partial t}-D \nabla^2 n(x,t) = 0$	& $\frac{\partial T(x,t)}{\partial t}-D_W \nabla^2 T(x,t) = 0$\\
								& $D_W \equiv \frac{K}{\rho c}, \rho:$ Massendichte, $c$: spez. W�rmekapazit�t\\\midrule
\multicolumn{2}{l}{Driftstromdichte: $j = n\cdot v_d \cdot f$\quad $f:$ ``generische'' Teilcheneigenschaft, z.B. $f = q$ f�r Ladung}\\
Teilchenstromdichte: $\boldsymbol{j(r)}$ 			& W�rmestromdichte: $\boldsymbol{j_Q(r)}\quad [\boldsymbol{j_Q(r)}] = \frac{J}{m^2\cdot s}$\\
Teilchenstrom: $I = \int_A \boldsymbol{j(r)}dA$			& W�rmestrom: $I_Q = \int_A \boldsymbol{j}_Q(\boldsymbol{r})dA$\\
Teilchendichte: $n(r,t)$					& W�rmedichte: $\rho_Q$
\end{tabular}
\subsection*{Beispiel station�rer Zustand}
W�rmestromdichte in 1D: $j_Q = -K \frac{\partial T(x)}{\partial x}$\\
$dT = -\frac{j_Q}{K}dx \Rightarrow \int_{T_1}^{T(x)}dT = -\frac{j_Q}{K}\int_{x_1}^{x}dx,$ falls $j_Q(r)$ konst.\\
Randbedingungen: $T(x_1) = T_1, T(x_2) = T_2$\\
W�rmestromdichte: $j_Q = -K \frac{T_2-T_1}{x_2-x_1}$\\
Temperaturprofil: $T(x) = T_1 -\frac{j_Q}{K}(x-x_1)$