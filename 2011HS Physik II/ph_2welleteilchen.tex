\subsection{Photoelektrischer Effekt}
\begin{tabular}{p{4cm} p{15cm}}
	  & \begin{tabular}[t]{l}
	    $\boxed{E_{kin,max} = \left(\tfrac{1}{2} mv^2 \right) = h\nu - \Phi_0}$\\
	    $E_{kin,max}:$ maximale kinetische Energie der austretenden Elektronen\\
	    $\Phi_0:$ Austrittsarbeit: Mindestenergie um ein Elektron herauszul�sen.
	    \end{tabular}\\
Grenzfrequenz	& $E_{kin,max} > 0 \Leftrightarrow h\nu_0 > \Phi_0 \Leftrightarrow \nu_0 > \frac{\Phi_0}{h}$
\end{tabular}
\subsection{Compton Effekt und Photonen}
\begin{tabular}{p{4cm} p{15cm}}
Nichtrelativistische Energie	& $E = \frac{p^2}{2m}$\\
Relativistische Energie	& $E = m(v)c^2 = c\sqrt{m_0^2c^2 + p^2}$\\
Geschw. abh�ngigkeit	& $m(v) = m_0 \cdot \frac{1}{\sqrt{1-\tfrac{c^2}{v^2}}}$\\
Photonen im Vakuum	& \begin{tabular}[t]{ll}
		   Ruhemasse	& $m_0 = 0$\\
        	   Energie	& $E = h\nu = \frac{hc}{\lambda} = cp\quad$ (nur f�r Teilchen mit $m_0 = 0$!!)\\
		   Impuls	& $p = \frac{h}{\lambda}$\\
		   Ausbreitungsgeschwindigkeit	& $c$
        	  \end{tabular}\\
Compton Effekt		& \begin{tabular}[t]{p{14cm}}
              		   $\Delta \lambda = \lambda_2 - \lambda_1 = \frac{h}{m_e c} (1-\cos\varphi)$\\
			   $\frac{h}{m_e c} =: \lambda_{\text{Compton}} = 2.426 \cdot 10^{-12}$m\\
			   $\lambda_1$: Wellenl�nge des Photons vor dem Stoss mit einem freien Elektron.\\
			   $\lambda_2$: Wellenl�nge nach dem Stoss.\\
			   $\varphi$: Winkel zwischen einfallendem Photon (Welle) und gestreutem Photon\\
			   Der Compton-Effekt hat nur dann einen merklichen Einfluss, falls $\Delta \lambda / \lambda_1$ gross ist, d.h. $\lambda_1$ muss im pm-Bereich liegen (R�ntgenstrahlen).
              		  \end{tabular}
\end{tabular}
\subsection{Welle-Teilchen Dualismus}
Elektromagnetische Strahlung breitet sich wie eine Welle aus, und wechselwirkt mit Materie wie ein Teilchen.\\
\begin{tabular}{p{4cm} p{15cm}}
Wellencharakteristiken:		& Transmission, Beugung, Reflexion, Interferenz\\
Teilchencharakteristiken:	& Absorption, Emission
\end{tabular}

\subsection{De Broglie-Hypothese}
Laut de Broglie haben auch Teilchen mit Ruhemasse $>$ 0 eine Wellenl�nge.\\
\begin{tabular}{p{4cm} p{15cm}}
Materiewellen	& \begin{tabular}[t]{ll}
                        Wellenl�nge	& $\lambda = \frac{h}{p}$\\
			nicht relativistischer Impuls	& $p = mv$\\
			relativistischer Impuls	& $p = \frac{m_0v}{\sqrt{1-(v/c)^2}}$\\
			Materiewelle	& $\Psi(z,t) = \Psi_0 e^{i(\omega t-kz)}\quad \omega = \frac{2\pi c}{\lambda}$
                  \end{tabular}
\end{tabular}