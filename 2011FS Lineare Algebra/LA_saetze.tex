\section*{Definitionen, S�tze, Lemmas und Korollare}
\section{Vektorr�ume}
\begin{tabular}{lp{15cm}}
 Satz 4.1	& Verschiedene Basen f�r einen Vektorraum bestehen aus gleich vielen Vektoren.\\
 Lemma 4.2	& Ein Erzeugendensystem f�r einen Vektorraum V besteht aus gleich vielen oder mehr Vektoren wie die Anzahl linear unabh�ngiger Vektoren von V.\\
 Satz 4.3	& Sei $V$ ein Vektorraum mit Dimension $n$.
		 \begin{itemize}
         	    \item Mehr als $n$ Vektoren in $V$ sind linear abh�ngig.
		    \item Weniger als $n$ Vektoren in $V$ sind nicht erzeugend.
		    \item $n$ Vektoren sind linear unabh�ngig, gdw. sie erzeugend sind, gdw. bilden sie eine Basis.
         	  \end{itemize}\\
 Satz 4.4	& In endlichdimensionalen Vektorr�umen gilt: Seien $||x||$ und $||x||'$ zwei Normen. Dann gibt es eine Konstante $c\geq 1$, so dass f�r jeden Vektor $x$ gilt: $\frac{1}{c} ||x||' \leq ||x|| \leq c||x||$\\
 Satz 4.5	& \begin{itemize}
         	   \item Die orthogonale Projektion eines Vektors $x$ auf den Vektor $y\neq 0$ ist gegeben durch den Vektor $\tfrac{\langle y,x \rangle}{\langle y,y \rangle} y$
		   \item $\forall x,y \in V: \langle x,y \rangle ^2 \leq \langle x,x \rangle \cdot \langle y,y \rangle$ (Schwarz'sche Ungleichung)
		   \item $||x|| := \sqrt{\langle x,x \rangle}$ ist eine Norm in V.
		   \item $\langle x,y \rangle = 0 \quad(x\bot y) \Rightarrow ||x+y||^2 = ||x-y||^2 = ||x||^2 + ||y||^2$
         	  \end{itemize}\\
 Satz 4.6	& In einem reellen n-dim Vektorraum sind die paarweise orthogonalen Einheitsvektoren $e^{(1)}, e^{(2)},...,e^{(k)}$ linear unabh�ngig.\\
 Korollar 4.7	& $n$ paarweise orthogonale Einheitsvektoren bilden eine orthonormale Basis in einem reellen n-dim VR\\
\end{tabular}


