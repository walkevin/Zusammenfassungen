\section{Normalformen}
\begin{tabular}{lp{15cm}}
 Satz 9.1	& \begin{itemize}
         	   \item Jede halbeinfache Matrix $A$ ist �hnlich zu einer Diagonalmatrix $D$
		   \item Jede reelle symmetrische Matrix $A$ ist orthogonal-�hnlich zu einer Diagonalmatrix $D$
		   \item Jede quadratische Matrix $A$ ist �hnlich zu einer Rechtsdreicksmatrix $R$ (In der Diagonalen von $R$ stehen die EW von $A$)
         	  \end{itemize}\\
 Def.		& Ein $\tilde{D}_{jj}$ der Ordnung 1 ist ein reeller EW von A\newline
		  Ein $\tilde{D}_{jj}$ der Ordnung 2 hat die Form $\tilde{D}_{jj} = \begin{pmatrix}
		                                                                 a_j & b_j\\
		                                                                -b_j & a_j\\
		                                                                \end{pmatrix}$, wobei $a_j \pm ib_j$ ein konjugiert komplexes Eigenwertpaar von $A$ ist.\\
 Satz 9.2	& Jede reelle halbeinfache quadratische Matrix $A$ ist �hnlich zu einer reellen Blockdiagonalmatrix\newline
		  $\tilde{D} = \begin{pmatrix}
		                \tilde{D}_{11}	& & &0 \\
				& \tilde{D}_{22}  & & \\
				& & \ddots          &\\
				0 & & &\tilde{D}_{kk}
		               \end{pmatrix}$ mit Matrizen $\tilde{D}_{jj}$ der Ordnung 1 oder 2.\\
		& Jede reelle quadratische Matrix $A$ ist �hnlich zu einer reellen oberen Blockdreiecksmatrix\newline
		  $\tilde{R} = \begin{pmatrix}
		                \tilde{R}_{11}	& & &\ast \\
				& \tilde{R}_{22}  & & \\
				& & \ddots          &\\
				0 & & &\tilde{R}_{kk}
		               \end{pmatrix}$ mit Matrizen $\tilde{R}_{jj}$ der Ordnung 1 oder 2.\\
 Satz 9.3	& Zu jeder reellen quadratischen Matrix $A$ existiert eine orthogonale Matrix $U$, sodass $\tilde{S} := U^T AU$ eine reelle obere Blockdreiecksmatrix ist:\newline
		  $R^{\ast} = \begin{pmatrix}
		                R^{\ast}_{11}	& &\ast \\
				& \ddots         &\\
				0 & & R^{\ast}_{kk}
		               \end{pmatrix}$ mit Matrizen $R^{\ast}_{jj}$ der Ordnung 1 oder 2. Die Ordnung 2 ist hier etwas anders:
Die EW-Paare stehen nicht direkt drin, sondern so, dass man sie zuerst mit dem char.Polynom berechnen muss.\\
\end{tabular}\\
\begin{tabular}{lp{15cm}}
  Satz 9.6	& {\bfseries Singul�rwertzerlegung} Zu jeder reellen $m\times n$-M. $A$ vom Rang $r$ existieren\newline
		  \begin{tabular}{lll}
		  eine orthogonale & $m\times m-$Matrix $U$ & $u^{(i)}$ = {\bfseries Links-Singul�rvektoren}\\
		  eine orthogonale & $n\times n-$Matrix $V$ & $v^{(i)}$ = {\bfseries Rechts-Singul�rvektoren}\\
		  eine		   & $m\times n-$Matrix $S$,& $s_i$ = {\bfseries Singul�rwerte} sodass
		  \end{tabular}\newline
		  \begin{equation*}
		   A = USV^T
		  \end{equation*}
		  $S$ hat dabei Diagonalgestalt, d.h. $S = \begin{cases}
		                                            \left(\tfrac{~\hat{S}~}{0}\right) & m \geq n\\
							    (\hat{S} | 0)		     & m \leq n
		                                           \end{cases}$ \newline und $\hat{S} = \mathrm{diag}(s_1,s_2,...,s_r,s_{r+1},...,s_p), p=\min(m,n)$\\
		& \begin{itemize}
		   \item $s_1 = ||A||_2, s_1 \geq s_2 \geq...\geq s_r > 0, s_{r+1} = ... = s_p = 0$
		   \item Die Zahlen $s^2_i$ sind die EW von $\begin{cases}
							      A^T A & m \geq n\\
							      A A^T & m \leq n
		                                            \end{cases}$
		   \item F�r die Spalten $u^{(i)}$ von $U$ und die Spalten $v^{(i)}$ von $V$ gilt:\newline
			  $Av^{(i)} = s_iu^{(i)}$\newline
			  $A^Tu^{(i)} = s_iv^{(i)},\quad i = 1,...,p$\newline
			  Die ``restlichen'' $v^{(i)}$ und $u^{(i)}$ sind dann gleich 0
		  \end{itemize}\\
 Korollar 9.7	& F�r jede $m\times n$-Matrix $A$ vom Rang $r$ gilt\newline
		  \begin{tabular}{lll}
		   Kern $A$	&=& span$\{v^{(r+1)},..,v^{(n)}\}$ = Spalten von $V$\\
		   Bild $A$	&=& span$\{u^(1),...,u^(r)\}$ = Spalten von $U$
		  \end{tabular}\\
		& F�r die transponierte Matrix $A^T$ gilt\newline
		  \begin{tabular}{lll}
		   Kern $A^T$	&=& span$\{u^{(r+1)},..,u^{(m)}\}$ = Spalten von $U$\\
		   Bild $A^T$	&=& span$\{v^(1),...,v^(r)\}$ = Spalten von $V$
		  \end{tabular}\\
 Satz 9.8	& Sei $A$ eine $m\times n$-Matrix. Dann\newline
		 $A = s_1E_1 + s_2E_2 +...+ s_pE_p,\quad p = \min(m,n)$\newline
		 wobei $E_i := u^{(i)}{v^{(i)}}^T$ mit Rang $E_i = 1, \sum_{k=1}^{m} \sum_{l=1}^n ((E_i)_{kl})^2 = 1$\\
 Lemma 9.9	& Sei $A$ eine $m\times n$-Matrix, $B$ eine $n\times p$-Matrix. Dann gilt f�r die $m\times p$-Produktematrix\\
		& $AB = a^{(1)}b^{[1]}+a^{(2)}b^{[2]}+...+a^{(n)}b^{[n]}$
\end{tabular}