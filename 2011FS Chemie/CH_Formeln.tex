\documentclass[a4paper,12pt,fleqn]{article}

\usepackage{CH_template}
\usepackage{color,colortbl}
\renewcommand{\arraystretch}{1.5}

\begin{document}


\section{Konstanten}
\begin{tabular}{l >{$}l<{$}}
 Stoffmenge 		& \n(X): 1\mol = 6.022\cdot 10^{23} \text{Teilchen} = \# \text{Atome in 12g $^{12}_6C$} \\
 molare Masse 		& \M(X): \g \cdot \mol^{-1} \quad \text{Bsp: }\M(H_2 O) = 2\cdot 1.0079 + 15.9994\\
 Masse 			& \m(X): \g\\
 \multicolumn{2}{l}{$\n(X) \cdot M(X) = m(X)$}\\
 Stoffmengenkonzentration & \M = \mol\cdot\mathrm{L}^{-1}\\
 Atommasseneinheit	& \mathrm{u} \quad 12\mathrm{u} = \text{Masse eines $^{12}_6 C$-Teilchens} = 1.66 \cdot 10^{-24}\g\cdot 12\\\hline

 Avogadro-Zahl		& \N_A = 6.022\cdot 10^{23} \mol^{-1}\\
 ideales Gasvolumen	& 22.41~ \mathrm{L} \cdot \mol^{-1}\\
 ideale Gaskonstante	& \R = 8.31447 \J\cdot\mol^{-1}\cdot\K^{-1} = 8.31447 \kPa\cdot\mathrm{L}\cdot\mol^{-1}\cdot\K^{-1}\\
 Grad Celsius		& ^{\circ}\C = \K-273.15\\
 Elementarladung	& \e = 1.6\cdot 10^{-19} \C\\
 Faraday-Konstante	& \F = \N_A \cdot \e = 96485 \C\cdot\mol^{-1}\\

\end{tabular}

\section{Geschwindigkeitsgesetz}
Allgemein: $v(AX) = k\cdot[A]^x\cdot[X]^y$ \\
``Die Reaktion ist x+y-ter Ordnung insgesamt, x-ter Ordnung bzgl. A und y-ter Ordnung bzgl. X''\\
\begin{tabular}{l|>{$}l<{$}|>{$}l<{$}|l|>{$}l<{$}}
 Ordnung	& \text{Geschw.gesetz}	& \text{Zeitabh. der Konzentration}	& lin. Beziehung	& \text{Halbwertszeit}\\\hline
 0		& v=k			& [A] = -k\cdot t+[A]_0			& [A] gegen t		& \frac{[A]_0}{2k}\\
 1		& v=k\cdot [A]		& \ln [A] = -k\cdot t+\ln [A]_0		& ln [A] gegen t	& \frac{\ln 2}{k}\\
 2		& v=k\cdot[A]^2		& \frac{1}{[A]} = k\cdot t + \frac{1}{[A]_0}&$\frac{1}{[A]}$ gegen t	& \frac{1}{k\cdot [A]_0}
\end{tabular}

\section{Massenwirkungsgesetz (MWG)}
 F�r die allgemeine Reaktion $aA + eE \rightleftharpoons xX + zZ$ lautet das MWG\\
$ \frac{[X]^x \cdot [Z]^z}{[A]^a \cdot [E]^e} = K_c = \frac{k_{\text{hin}}}{k_{\text{r�ck}}}\quad\text{In mehrstufigen Reakt. ist K das Produkt der Einzelreakt.}\\
K_p = \frac{p^x(X) \cdot p^z(Z)}{p^a(A) \cdot p^e(E)},\quad\text{falls \textbf{alle} beteiligten Stoffe Gase sind.}\\
K_p = K_c \cdot (RT)^{\Delta n},\qquad \begin{array}{l}
                                         R = \text{ideale Gaskonstante}\\
					 T = \text{Temperatur }[K]\\
					 \Delta n = x+z-(a+e)
                                      \end{array}\\
$
\section{S�ure-Base}
\begin{tabular}{l >{$}l<{$}}
 allg.Reakt.gleichung	& HA + H_2 O \rightleftharpoons A^- + H^+\\
			& B^- + H_2 O \rightleftharpoons B + OH^-\\
 $pH$			& pH = -\log [H^+] \mathrm{L}\cdot\mol^{-1}\\
 $pOH$			& pOH = -\log [OH^-] \mathrm{L}\cdot\mol^{-1}\\
 $pK_w$			& pK_w = pH + pOH = 14\\
 Ionenprodukt 		& 10^{-14} = [OH^-][H^+] \\
 Massenerhaltung	& [HA]_0 = [HA] + [A^-]\\
 Elektroneutralit�t	& [H+] = [OH^-] + [A^-]\\
 S�uredissoziationskonstante	& K_s(HA) = \frac{[A^-] \cdot [H^+]}{[HA]}\\
 $pK_s$			& pK_s = -\log K_s\\
 S�urest�rke		& \begin{array}{ll}
            		   \text{schwach:}	& pK_s > 3.5\\
			   \text{mittel:}	& 3.5 > pK_s > 0.35\\
			   \text{stark:}	& pK_s < 0.35
            		  \end{array}\\
 Mehrprotonige S�uren	& pK_2 = pK_1 + 5, pK_3 = pK_2 + 5,\quad \text{Bsp: }K_1 = \frac{[H_2 A^-][H^+]}{[H_3 A]}\\
 Oxidos�uren		& \text{F�r Oxidos�uren } H_a XO_b \text{ gilt:}\\
			& \begin{array}{lll}
			   pK_1 = 7	& \text{falls }a=b	&HOCl\\
			   pK_1 = 2	& \text{falls }a=b-1	&(HO)_2 SO\\
			   pK_1 = -3	& \text{falls }a=b-2	&HONO_2\\
			  \end{array}\\
 Basendissoziationskonstante	& K_b(A^-) = \frac{[OH^-] \cdot [HA]}{[A^-]}\\
 $pK_b$			& pK_b = -\log K_b\\
 $pK_w$			& pK_w = pK_s + pK_b = 14\quad\text{Bsp: }pK_s(NH_4^+) + pK_b(NH_3) = 14\\
 Dissoziationsgrad	& \alpha = \frac{[A]^-}{[HA]}\\
 Zsh. $K_s$ mit $[H^+]$	& [H^+] = -\frac{1}{2} K_s + \sqrt{\frac{1}{4}K_s^2+K_s\cdot c_0},\quad c_0 = \text{Anfangskonz. von } HA\\
			& \text{mit }HA \rightleftharpoons H^+ + A^- \\
 $[H^+]$ schwacher S�uren& [H^+] = \sqrt{K_s \cdot c_0}\\
 $[OH^-]$ schwacher Basen& [OH^-] = \sqrt{K_b \cdot c_0}\\
 $pH$ schwacher S�uren	& pH \approx \frac{1}{2}\left(pK_s - \log\left(\frac{c_0}{\mol \cdot \mathrm{L}^{-1}}\right)\right)\\
 $pOH$ schwacher Basen	& pOH \approx \frac{1}{2}\left(pK_b - \log\left(\frac{c_0}{\mol \cdot \mathrm{L}^{-1}}\right)\right)\\
 H.son-Hasselbalch-Gleichung & \text{Charakteristik von Pufferl�sungen}\\
			& pH = pK_s(HA) - \log \frac{[HA]}{[A^-]}, \quad\text{wobei } \frac{[HA]}{[A^-]} \in \left[\tfrac{1}{10},10\right]\\
 HSAB-Regel		& \text{Harte s�ure lieben harte Basen, weiche S�uren weiche Basen}\\
 Neutralisation		& V_{HA}\cdot[HA] = V_B\cdot[B]\\
 Metall-Ionen als S�ure	& [M(OH_2)_x]^{n+} + H_2 O \rightleftharpoons [M(OH_2)_{x-1}(OH)]^{n-1} + H_3 O^+
 
\end{tabular}

\section{L�slichkeitsprodukt}
 F�r die allgemeine Reaktion $A_aX_x \rightleftharpoons aA^{x+} + xX^{a-}$ lautet das L�slichkeitsprodukt\\
 $L = \left[A^{x+}\right]^a \cdot \left[X^{a-}\right]^x$\\
\subsection*{F�llungsreaktionen}
$
I < L \Rightarrow \text{L�sung unges�ttigt}\\
I = L \Rightarrow \text{L�sung ges�ttigt}\\
I > L \Rightarrow \text{L�sung �bers�ttigt, es gibt eine F�llung}\\
$
\begin{itemize}
 \item $\left[H^+\right]^2 \cdot \left[S^{2-}\right] = 1.1\cdot 10^{-22}$\\
 \item vergr�sserte Volumina $\Rightarrow$ verringerte Konzentrationen\\
 \item pH-Angabe $\Rightarrow [OH^-]$ resp. $[H^+]$-Konz. anpassen
\end{itemize}

\section{Grundlagen der chemischen Thermodynamik}

\subsection*{1.Hauptsatz (Energieerhaltung)}
$
\Delta H = \Delta U + p\Delta V\qquad\begin{array}{l}
                                     \Delta H = \text{Reaktionsenthalpie}\\
				     \Delta U = U_2-U_1, \text{Ver�nderung der inneren Energie, W�rme, Reakt.energie bei }\Delta V = 0\\
				     \Delta V = \text{Ver�nderung des Volumens}\\
				     p = \text{Atmosph�rendruck}\\
                                    \end{array}\\
\text{Enthalpie f�r Reaktionen mit Gasen}\\
\Delta H = \Delta U + \Delta n RT\qquad\begin{array}{l}
                                       \Delta n = \text{Ver�nderung der Stoffmenge}\\
					RT = 2.479\frac{kJ}{\mol}\enspace\text{@ 25\textcelsius}\\
                                      \end{array}
$
\subsection*{2.Hauptsatz}
\begin{itemize}
 \item Bei einer spontanen Zustands�nderung vergr�ssert sich die Entropie S. S ist ein Mass f�r die Unordnung eines Systems.\\
 \item Bei einer Reaktion wird ein Energieminimum angestrebt.\\
 \item Bei einer Reaktion wir ein Maximum an Unordnung angestrebt.\\
\end{itemize}
\begin{tabular}{l >{$}l<{$}}
 Gesamtentropie		& \Delta S_{Ges} = \Delta S_{Sys} + \Delta S_{Umg}\\
			& \Delta S_{Umg} = -\frac{\Delta H}{T}\\
 Gibbs-Energie (kJ/mol)	& \Delta G = \Delta H - T\Delta S\\
 Gibbs-freie Enthalpie	& G = H-TS\\
 Ablauf der Reaktion	& \begin{array}{ll}
                    	   \Delta G < 0	& \text{Reaktion verl�uft spontan}\\
			   \Delta G = 0	& \text{System  in Glgw.}\\
			   \Delta G > 0	& \text{Reaktion verl�uft nicht spontan}
                    	  \end{array}\\
 freie Reaktionsenergie	& \Delta F = \Delta U - T\Delta S\\
 Standard-Bildungsenthalpie	& \Delta G_f^{\circ} = \text{Gibbs-Energie f�r 1mol der Verbindung}\\
			& \Delta G^{\circ}_f = \Delta H^{\circ}_f - T\Delta S^{\circ}\\
			& \Delta G^{\circ} = \sum \Delta G_f^{\circ}\text{Produkte} - \sum \Delta G_f^{\circ}\text{Edukte}\\
			& \Delta G^{\circ}(\text{Element}) \equiv 0

\end{tabular}

\subsection*{3.Hauptsatz}
\begin{itemize}
 \item Die Entropie einer Substanz nimmt mit der Temperatur zu.\\
 \item Die Entropie einer perfekten kristallinen Substanz beim absolut Nullpunkt ist 0
\end{itemize}
\begin{tabular}{l >{$}l<{$}}
 absolute Standard-Entropie	& S^0: \text{Entropie bei }101.3kPa\\
 Standard-Reaktionsentropie	& \Delta S^0: \sum\text{abs.Entropie(Produkte)} - \sum\text{abs.Entropie(Edukte)}\\
 Temperaturabh�ngigkeit von K	& \Delta G^0 = -RT\cdot\ln K\quad\text{K = K aus MWG}\\
				& G = G^0 + RT\cdot\ln a\quad a = \frac{p}{101.3kPa} = f\cdot\frac{c}{\mol\cdot\mathrm{L}^{-1}}\\
				& RT = 2.479 \frac{kJ}{mol}\\
 Clausius-Clap.-Glgw.		& \ln\left(\frac{p_2}{p_1}\right) = \frac{\Delta H_v}{R}\left(\frac{1}{T_1}-\frac{1}{T_2}\right)\\
 CCG II				& \ln\left(\frac{K_2}{K_1}\right) = \frac{\Delta H^0}{R}\left(\frac{1}{T_1}-\frac{1}{T_2}\right)\\
\end{tabular}

\section{Elektrochemie}
\subsection*{Masseinheiten und Beziehungen}
\begin{tabular}{lll}
			 & Symbol		& Masseinheit\\
elektr.Potential	 & E			& Volt (V)\\
Stromst�rke		 & I			& Amp�re (A)\\
Widerstand		 & R			& Ohm ($\Omega$)\\
Ladung			 & q,e,L		& Coulomb (C)\\
Leitf�higkeit		 & $\frac{1}{R}$	& Siemens (S)\\\hline
Elementarladung		 & $\e = 1.6\cdot 10^{-19} \C$&\\
Faraday-Konstante	 & $\F = \N_A \cdot \e = 96485 \C\cdot\mol^{-1} = 96.485 \frac{kJ}{V\cdot \mol}$&\\
\end{tabular}\\
$1A = \frac{1C}{s}\qquad		1J = 1V\cdot C\qquad	1\Omega = 1\frac{V}{A} = 1J\cdot s\qquad		1S = 1\Omega^{-1}$\\

\subsection*{Faraday'sches Gesetz}
$m = \frac{M}{z}\cdot\frac{L}{F}\qquad\begin{array}{l}
                                      m = \text{abgeschiedene Menge} [kg]\\
				      \frac{M}{z} = \text{molare �quivalentmasse} [kg]\\
					\quad M = \text{molare Masse}\\
					\quad z = \text{Anzahl mol Elektronen}\\
				      L = \text{Elektrizit�tsmenge} [C = A\cdot s]\\
				      F = 96.485 \frac{kJ}{V\cdot \mol}
                                     \end{array}
$

\subsection*{Galvanische Zelle}
\begin{tabular}{l >{$}l<{$}}
 Erzeugte elektr. Energie	& W = L\cdot\Delta E\\
 Gibbs-Energie (kJ/mol)		& \Delta G = -\underbrace{n\cdot F}_{L} \cdot \Delta E \quad\text{n = Anz. beteiligte Elektr. bei Reaktion}\\
 Spannung			& \Delta E = E^0(Kathode)-E^0(Anode)\quad\text{Stromgewinn bei $\Delta E > 0$}\\
 Nernst-Gleichung		& \Delta E = \Delta E^0 - \frac{2.303~ RT}{nF}\log Q\quad \begin{array}{l}
                 		                                                        Q = \frac{a^x(X)\cdot a^z(Z)}{a^a(A)\cdot a^e(E)}\\
											\text{Feststoffe: } a = [~]\\
											\text{Gase: } a = \frac{p}{101.3kPa}
                 		                                                       \end{array}\\
 ~~@ 25 \textcelsius	& \Delta E = \Delta E^0 - \frac{0.05916}{n}\log Q\\
 ~~Halbreaktion			& E = E^0 + \frac{0.05916}{n}\log \frac{[Oxidiert]}{[Reduziert]}
\end{tabular}


\end{document}
