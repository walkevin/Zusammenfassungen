\documentclass[a4paper,12pt,fleqn]{article}

\usepackage{CH_template}
\usepackage{color,colortbl}
\renewcommand{\arraystretch}{1.5}

\begin{document}
 \section{Nomenklatur}
  \subsection{Allgemeine Regeln}
  \begin{itemize}
   \item Sortierung nach Alphabet.\\
   \item Sortierung nach Elektronegativit�t. Negativeres zuerst.\\
   \item Salze: Kation (+ Ion) zuerst.\\
   \item Zentralatom zuletzt.
  \end{itemize}

  \subsection{Zahlenpr�fixe}
  \begin{tabular}{lll}
   Anzahl	& Pr�fix	& Alternativ\\
   2		& di		& bis\\
   3		& tri		& tris\\
   4		& tetra		& tetrakis\\
   5		& penta		& pentakis\\
   6		& hexa		& hexakis\\
   7		& hepta		& heptakis\\
   8		& octa&\\
   9		& nona&\\
   10		& deca&\\
   11		& undeca&\\
   12		& dodeca&\\
  \end{tabular}

  \subsection{Pr�fixe spezieller Elemente}
  C: carb(o)	N: nitr(o)	O: ox(ygenium)	S: sulf(ur)	Fe: ferr(um)

  \subsection{Anionen der Wasserstoffs�uren}
  \begin{tabular}{l >{$}l<{$}}
   Kaliumhexafluoridoantimonat	& KSbF_6\\
   Schwefelhexafluorid		& SF_6\\
   Natriumsulfid		& Na_2S\\

  \end{tabular}

  \subsection{Anionen der Sauerstoffs�uren}
  \begin{tabular}{l >{$}l<{$}}
   Chlorat			& ClO_3^-\\
   Sulfat			& SO_4^{2-}\\
   Carbonat			& CO_3^{2-}\\
   Kaliumhydroxidotrioxidosulfat& KHSO_4\\
   ??Bariumbis(hydridotrioxidosulfat)	& Ba(HSO_3)_2\\
  \end{tabular}

  \subsection{Radikale}
  \begin{tabular}{l >{$}l<{$}}
   Hydroxyl	& HO^{\cdot -}\\
   Mehtyl	& CH_3^{\cdot -}\\
   Dioxidat			& O_2^{\cdot -}\\
   Oxidostickstoff(\textbullet)	& NO^{\cdot}\\
   Dioxidonitrat(\textbullet 2-)& NO_2^{\cdot 2-}\\
  \end{tabular}

  \subsection{Organische Chemie}
  \begin{tabular}{l >{$}l<{$}}
   Propan-1,2-diol	& CH_3-CHOH-CH_2OH\\
   2-Buten-1-ol		& CH_3-CH_2=CH_2-CH_2OH\\
   Pentaaz-2-en			& NH_2-N=NH-NH-NH_2\\
   Azanol, Hydroxylamin		& NH_2OH\\
   Dihydroxidooxidokohlenstoff	& CO(OH)_2\\
   Hydridohydroxidooxidokohlenstoff	& HCOOH\\
   Hydridobrom			& HBr\\
   Hydroxidodioxidobrom		& HBrO_3\\
  \end{tabular}

  \subsection{Sonstige}
  \begin{tabular}{l >{$}l<{$}}
   Dihydridosauerstoff		& H_2 O\\
   Dioxidokohlenstoff		& CO_2\\
   Bis(dioxidosulfat)(S-S)(2-)	& (SO_2)_2^{2-}\\
   Trichloridoeisen		& FeCl_3\\
   Kaliumhexacyanidoferrat(4-)	& K_4\left[Fe(CN)_6\right]\\
   Triammintrichloridoplatinchlorid	& \left[ Pt(NH_3)_3 Cl_3\right] Cl\\
   Kaliumamminpentachloridoplatinat	& K\left[ Pt(NH_3) Cl_5\right]\\
   Nitridosulfidocarbonate(1-)	& CNS^-
  \end{tabular}

\end{document}
