\section{Punktsch�tzungen}
\subsection{Momentenmethode}
\begin{tabular}{p{4cm} >{$}p{16cm}<{$}}
Parametervektor $\Theta$	& \text{Enth�lt die f�r eine Verteilung relevanten Parameter. Bsp: } \Theta = (\mu, \sigma^2) \text{ bei Normalverteilung}, \Theta = \lambda \text{ bei Poissonverteilung}\\
Sch�tzer f�r $\Theta$		& \hat{\Theta}_j := g_j(\hat{\mu}_1,...,\hat{\mu}_d)\quad j = 1,...,d\\
				& \text{wobei } \hat{\mu}_j := \frac{1}{n} \sum_{i=1}^n x_i^j = \E(X^k)\quad j = 1,...,d\\
Bsp: Poissonverteilung		& \begin{array}[t]{lll}
                      		  \lambda = \E(X)			& \Rightarrow	& \hat{\lambda} = \frac{1}{n} \sum_{i=1}^n X_i\\
				  \lambda = \V(x) = \E(X^2) - \E(X)^2	& \Rightarrow
                      		  \end{array}
\end{tabular}

\subsection{Maximum-Likelihood-Sch�tzer}
\begin{tabular}{p{4cm} >{$}p{16cm}<{$}}
Max.Likelihood Funktion		& L(\vec \Theta ; \vec x) = \prod_{i=1}^n f_{\Theta}(x_i)\quad f_{\Theta} = \text{Dichtefunktion}\\
log-Likelihood Fkt.		& \ell := \log (L) = \sum_{i=1}^n \log f_{\Theta}(x_i)\\
\end{tabular}
\begin{tabular}{p{4cm} p{15cm}}
Vorgehen	& \begin{enumerate}
        	  	\item Stelle Max.Likelihood Fkt. f�r die gew�nschte Verteilung auf. Die Dichtefunktion soll dabei mit den Parametern $\Theta_j$ geschrieben werden. Bsp: $f(x) = \lambda e^{-\lambda x} \Leftrightarrow f(\Theta ; \vec x) = \Theta e^{-\Theta x_i}$. Bei mehreren Parametern entsprechend mit $\Theta_1, \Theta_2$ usw.
			\item Berechne $\Theta_j$ f�r die $L(\vec \Theta; \vec x)$ oder $\ell(\vec \Theta; \vec x)$ maximal wird.
        	  \end{enumerate}
\end{tabular}
\subsubsection{Beispiel}
$X \sim EXP(\Theta)$\\
\begin{tabular}{p{4cm}  >{$}p{16cm}<{$}}
Dichtefunktion		& f(\Theta ; \vec x) = \Theta e^{-\Theta x_i}\\
$L(\Theta ; \vec x)$	& L(\Theta ; \vec x) = \prod_{i=1}^n \Theta e^{-\Theta x_i}\\
$\ell(\Theta ; \vec x)$	& \ell(\Theta ; \vec x) = \sum_{i=1}^n \ln \left( \Theta e^{-\Theta x_i} \right) = \sum_{i=1}^n \ln (\Theta) - \Theta x_i\\
Ableiten		& 0 = \frac{\partial \ell}{\partial \Theta} = \sum_{i=1}^n \frac{1}{\Theta} - x_i = \frac{n}{\Theta} - \sum_{i=1}^n x_i\\
Sch�tzer		& \Rightarrow \hat{\Theta} = \frac{n}{\sum_{i=1}^n x_i} = \frac{1}{\widebar{X}_n}
\end{tabular}