\section{Vergleich zweier Stichproben}
\subsection{Randomisierung (Ungepaarte Vergleiche)}
Bsp: $X_i:$ 60 Probanden getestet mit Medikament, $Y_i:$ 40 Probanden getestet mit Placebo. (Verschiedene Testbedingungen)
\subsubsection{Zwei-Stichproben t-Test}
\begin{tabular}{p{4cm} >{$}p{16cm}<{$}}
Annahme		& X_1,...,X_N \sim^{iid} N(\mu_X,\sigma^2)\\
		& Y_1,...,Y_N \sim^{iid} N(\mu_Y,\sigma^2)\\
Nullhypothese	& \begin{array}[t]{l|l}
    		  H_0: \mu_X = \mu_Y	& H_A: \mu_X > \mu_Y\\
		  H_0: \mu_X = \mu_Y	& H_A: \mu_X < \mu_Y\\
		  H_0: \mu_X = \mu_Y	& H_A: \mu_X \neq \mu_Y
    		  \end{array}\\
Teststatistik	& t:= \frac{\widebar{X}_n - \widebar{Y}_m}{S_{pool}\sqrt{\frac{1}{n} + \frac{1}{m}}}\\
		& \text{wobei } S_{pool} = \frac{1}{n+m-2} \left( \sum_{i=1}^n (X_i - \widebar{X}_n)^2 + \sum_{i=1}^m (Y_i - \widebar{Y}_m)^2 \right)\\
		& \text{Verwerfe $H_0$ auf Niveau $\alpha$, falls}\\
		& \begin{array}[t]{lcl|l|l}
		  t_{n-1}	& \geq	& T_{n+m-2}(1-\alpha)				& VB_{\alpha} = [T_{n+m-2}(1-\alpha), \infty[	& \text{f�r }H_A: \mu_X > \mu_Y\\
		  t_{n-1}	& \leq	& T_{n+m-2}(\alpha)				& VB_{\alpha} = ]-\infty, T_{n+m-2}(\alpha)]	& \text{f�r }H_A: \mu_X < \mu_Y\\
		  |t_{n-1}|	& \geq	& T_{n+m-2}\left(1-\frac{\alpha}{2} \right)	& VB_{\alpha} = ]-\infty, T_{n+m-2}(\tfrac{\alpha}{2})] \cup 	& \text{f�r }H_A: \mu_X \neq \mu_Y\\
				&	&						& [T_{n-1}(1-\tfrac{\alpha}{2}), \infty[
		  \end{array}\\
\end{tabular}
\subsection{Blockbildung (Gepaarte Vergleiche)}
Bsp: $X_i:$ Reifentyp 1, getestet unter Bedingung A, $Y_i:$ Reifentype 2, getestet unter gleichen Bedingungen A\\
Neue Zufallsvariable: $\boxed{U_i = X_i - Y_i}$\\
\subsubsection{t-Test}
Anwenden, falls $U_i$ normalverteilt.
\subsubsection{Vorzeichentest}
Ausser i.i.d keine Voraussetzungen.\\
\begin{tabular}{p{4cm} >{$}p{16cm}<{$}}
Teststatistik		& V = \sum_{i=1}^n I_{u_i > 0}\quad \text{Anzahl positive $u_i$ f�r $i=1,...,n$}\\
			& V \sim BIN (n,p), p = \P(U_i > 0)\\
Nullhypothese		& H_0: p = \frac{1}{2}\\
Test			& \text{Binomialtest}
\end{tabular}
\subsubsection{Wilcoxontest}
\begin{tabular}{p{4cm} p{15cm}}
Teststatistik		& $W = \sum_{i=1}^n \mathrm{Rang}(|U_i|) \cdot I_{U_i > 0}$\\
Rang			& Die $U_i$ werden aufsteigend nach Absolutwert geordnet. Das absolut kleinste $U_i$ erh�lt Rang 1, dann Rang 2 usw.\\
			& Bsp: $U = \{-4,0,5\}: \quad |U_2| = 0 < |U_1| = 4 < |U_3| = 5 \Rightarrow W = 2 \cdot 0 + 1 \cdot 1 + 3 \cdot 1 = 4$\\
Nullhypothese		& Median $(U_i) = 0$\\
Test			& F�r gegebenes $n, \alpha, H_A$ schaue in Wilcoxon-Tabelle. Falls $W \in $ Intervall, verwerfe $H_0$
\end{tabular}


