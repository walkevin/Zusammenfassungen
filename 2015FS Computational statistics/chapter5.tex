\section{Bootstrap}
\subsection{Bootstrap algorithm}
\begin{application}
\begin{itemize}
 \item Useful for making statistical inference (confidence intervals, mean, variance, etc.)
 \item[\leftthumbsup] Bootstrap not only valid for large $n$, as opposed to central limit theorem.
\end{itemize}
\end{application}
\begin{theory}
 Given: data set $\{Z_1, \cdots, Z_n\} \text{ i.i.d.} \sim P$, $P$ unknown. $Z_i$ may be a vector.
 \begin{enumerate}
  \item Draw $n$ uniform random samples with replacement (dt. mit Zur{\"u}cklegen) from data set to yield bootstrap sample
        \begin{equation*}
         \{Z_1^*, \cdots, Z_n^*\}
        \end{equation*} 
  \item Compute bootstrapped estimator (e.g. mean)
        \begin{equation*}
         \hat{\theta}_n^{\ast} = g(Z_1^*, \cdots, Z_n^{\ast})
        \end{equation*}
  \item Repeat steps 1 \& 2 $B$ times to get
        \begin{equation*}
         \hat{\theta}_1^{\ast}, \cdots, \hat{\theta}_n^B
        \end{equation*}
  \item Compute approximate bootstrap expectation, variance and quantiles for the estimator
        \begin{align*}
        \mathbb{E}[\hat{\theta}^{\ast}_n] & \approx \frac{1}{B} \sum_{i=1}^B \hat{\theta}_n^{\ast i}\\
        \Var^{\ast}                       & \approx \frac{1}{B-1} \sum_{i=1}^B \left( \hat{\theta}_n^{\ast i} - \mathbb{E}[\hat{\theta}^{\ast}_n] \right)^2
        \end{align*}  
        The quantiles are just the empirical quantiles of $\hat{\theta}_1^{\ast}, \cdots, \hat{\theta}_n^B$
 \end{enumerate}
\end{theory}
\begin{code}
 # Define estimator
 g <- function(ind, data){ # ind: n x 1 matrix
    bst.smp <- data[ind,] # Draw a bootstrap sample
    # Do anything with the sample, e.g. mean(bst.smp)
    # Return a p x 1 matrix
 }
 
 # Create a n x B matrix of random indices
 n <- nrow(data) # Assume one row corresponds to one datapoint Z_i
 ind <- replicate(B, sample.int(n, replace = TRUE)) # n x B matrix
 
 # Compute bootstrapped estimators (of dimension p)
 bst.est <- apply(ind, 2, g, data = data) # p x B matrix
 
 # Compute expectation, variance, quantiles
 bst.ex  <- apply(bst.est, 1, mean) # p x 1 matrix
 bst.var <- apply(bst.est, 1, var)  # p x 1 matrix
 alpha <- 0.1
 bst.qt  <- apply(bst.est, 1, quantile, probs=c(alpha/2, 1-alpha/2)) # p x 2 matrix

 # Caution! The quantiles do not coincide with the confidence interval.
 \end{code}

 