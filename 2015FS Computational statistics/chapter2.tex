\section{Nonparametric Density Estimation}
\subsection{Kernel estimator}
\begin{theory}
Similar to a histogram, the density function $f(\cdot)$ at a point $x$ can be represented as
\begin{equation*}
 f(x) = \lim_{h\to 0} \frac{1}{2h} \mathbb{P}[x-h < X \leq x+h]
\end{equation*}
The probability can be estimated using a kernel function
\begin{gather*}
 \hat{f}(x) = \frac{1}{nh} \sum_{i=1}^n K \left( \frac{x - X_i}{h} \right) \\
 K(x) \geq 0, \int_{-\infty}^{\infty} K(x) dx = 1, K(x) = K(-x)
\end{gather*}
$h$ is the bandwidth and is a tuning parameter. Popular kernels are the Gaussian kernel
\begin{equation*}
 K_G(x) = \sqrt{2\pi} e^{-x^2 / 2}
\end{equation*}
and the Epanechnikov kernel
\begin{equation*}
 K_E(x) = \frac{3}{4} \left( 1 - |x|^2 \right) \mathbf{1} (|x| \leq 1)
\end{equation*}
\end{theory}
\begin{code}
 ke <- density(data, bw = 0.2, kernel="gaussian", n = , from = , to =)
\end{code}


\subsection{Bias-Variance trade-off}
\begin{align*}
 \text{Bias} &= \mathbb{E}[\hat{f}(x)] - f(x)\\
 \text{Variance} &= \mathbb{E}[\hat{f}(x)^2] - \mathbb{E}[\hat{f}(x)]^2
\end{align*}

\framebox{The bias of $\hat{f}$ increases and the variance of $\hat{f}$ decreases as $h$ increases}
For example, if $h$ was infinite, $\hat{f}(x)$ would be a constant, thus it had no variance but a big bias.

\subsection{The curse of dimensionality}
It has been shown that the best possible MSE rate is
\begin{equation*}
 \mathcal{O}(n^{-4/(4+d)})
\end{equation*}
